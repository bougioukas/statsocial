% Options for packages loaded elsewhere
\PassOptionsToPackage{unicode}{hyperref}
\PassOptionsToPackage{hyphens}{url}
\PassOptionsToPackage{dvipsnames,svgnames,x11names}{xcolor}
%
\documentclass[
  letterpaper,
  DIV=11,
  numbers=noendperiod,
  oneside]{scrreprt}

\usepackage{amsmath,amssymb}
\usepackage{iftex}
\ifPDFTeX
  \usepackage[T1]{fontenc}
  \usepackage[utf8]{inputenc}
  \usepackage{textcomp} % provide euro and other symbols
\else % if luatex or xetex
  \usepackage{unicode-math}
  \defaultfontfeatures{Scale=MatchLowercase}
  \defaultfontfeatures[\rmfamily]{Ligatures=TeX,Scale=1}
\fi
\usepackage{lmodern}
\ifPDFTeX\else  
    % xetex/luatex font selection
\fi
% Use upquote if available, for straight quotes in verbatim environments
\IfFileExists{upquote.sty}{\usepackage{upquote}}{}
\IfFileExists{microtype.sty}{% use microtype if available
  \usepackage[]{microtype}
  \UseMicrotypeSet[protrusion]{basicmath} % disable protrusion for tt fonts
}{}
\makeatletter
\@ifundefined{KOMAClassName}{% if non-KOMA class
  \IfFileExists{parskip.sty}{%
    \usepackage{parskip}
  }{% else
    \setlength{\parindent}{0pt}
    \setlength{\parskip}{6pt plus 2pt minus 1pt}}
}{% if KOMA class
  \KOMAoptions{parskip=half}}
\makeatother
\usepackage{xcolor}
\usepackage[left=1in,marginparwidth=2.0666666666667in,textwidth=4.1333333333333in,marginparsep=0.3in]{geometry}
\setlength{\emergencystretch}{3em} % prevent overfull lines
\setcounter{secnumdepth}{5}
% Make \paragraph and \subparagraph free-standing
\makeatletter
\ifx\paragraph\undefined\else
  \let\oldparagraph\paragraph
  \renewcommand{\paragraph}{
    \@ifstar
      \xxxParagraphStar
      \xxxParagraphNoStar
  }
  \newcommand{\xxxParagraphStar}[1]{\oldparagraph*{#1}\mbox{}}
  \newcommand{\xxxParagraphNoStar}[1]{\oldparagraph{#1}\mbox{}}
\fi
\ifx\subparagraph\undefined\else
  \let\oldsubparagraph\subparagraph
  \renewcommand{\subparagraph}{
    \@ifstar
      \xxxSubParagraphStar
      \xxxSubParagraphNoStar
  }
  \newcommand{\xxxSubParagraphStar}[1]{\oldsubparagraph*{#1}\mbox{}}
  \newcommand{\xxxSubParagraphNoStar}[1]{\oldsubparagraph{#1}\mbox{}}
\fi
\makeatother


\providecommand{\tightlist}{%
  \setlength{\itemsep}{0pt}\setlength{\parskip}{0pt}}\usepackage{longtable,booktabs,array}
\usepackage{calc} % for calculating minipage widths
% Correct order of tables after \paragraph or \subparagraph
\usepackage{etoolbox}
\makeatletter
\patchcmd\longtable{\par}{\if@noskipsec\mbox{}\fi\par}{}{}
\makeatother
% Allow footnotes in longtable head/foot
\IfFileExists{footnotehyper.sty}{\usepackage{footnotehyper}}{\usepackage{footnote}}
\makesavenoteenv{longtable}
\usepackage{graphicx}
\makeatletter
\def\maxwidth{\ifdim\Gin@nat@width>\linewidth\linewidth\else\Gin@nat@width\fi}
\def\maxheight{\ifdim\Gin@nat@height>\textheight\textheight\else\Gin@nat@height\fi}
\makeatother
% Scale images if necessary, so that they will not overflow the page
% margins by default, and it is still possible to overwrite the defaults
% using explicit options in \includegraphics[width, height, ...]{}
\setkeys{Gin}{width=\maxwidth,height=\maxheight,keepaspectratio}
% Set default figure placement to htbp
\makeatletter
\def\fps@figure{htbp}
\makeatother
% definitions for citeproc citations
\NewDocumentCommand\citeproctext{}{}
\NewDocumentCommand\citeproc{mm}{%
  \begingroup\def\citeproctext{#2}\cite{#1}\endgroup}
\makeatletter
 % allow citations to break across lines
 \let\@cite@ofmt\@firstofone
 % avoid brackets around text for \cite:
 \def\@biblabel#1{}
 \def\@cite#1#2{{#1\if@tempswa , #2\fi}}
\makeatother
\newlength{\cslhangindent}
\setlength{\cslhangindent}{1.5em}
\newlength{\csllabelwidth}
\setlength{\csllabelwidth}{3em}
\newenvironment{CSLReferences}[2] % #1 hanging-indent, #2 entry-spacing
 {\begin{list}{}{%
  \setlength{\itemindent}{0pt}
  \setlength{\leftmargin}{0pt}
  \setlength{\parsep}{0pt}
  % turn on hanging indent if param 1 is 1
  \ifodd #1
   \setlength{\leftmargin}{\cslhangindent}
   \setlength{\itemindent}{-1\cslhangindent}
  \fi
  % set entry spacing
  \setlength{\itemsep}{#2\baselineskip}}}
 {\end{list}}
\usepackage{calc}
\newcommand{\CSLBlock}[1]{\hfill\break\parbox[t]{\linewidth}{\strut\ignorespaces#1\strut}}
\newcommand{\CSLLeftMargin}[1]{\parbox[t]{\csllabelwidth}{\strut#1\strut}}
\newcommand{\CSLRightInline}[1]{\parbox[t]{\linewidth - \csllabelwidth}{\strut#1\strut}}
\newcommand{\CSLIndent}[1]{\hspace{\cslhangindent}#1}

\usepackage{booktabs}
\usepackage{longtable}
\usepackage{array}
\usepackage{multirow}
\usepackage{wrapfig}
\usepackage{float}
\usepackage{colortbl}
\usepackage{pdflscape}
\usepackage{tabu}
\usepackage{threeparttable}
\usepackage{threeparttablex}
\usepackage[normalem]{ulem}
\usepackage{makecell}
\usepackage{xcolor}
\KOMAoption{captions}{tableheading}
\makeatletter
\@ifpackageloaded{tcolorbox}{}{\usepackage[skins,breakable]{tcolorbox}}
\@ifpackageloaded{fontawesome5}{}{\usepackage{fontawesome5}}
\definecolor{quarto-callout-color}{HTML}{909090}
\definecolor{quarto-callout-note-color}{HTML}{0758E5}
\definecolor{quarto-callout-important-color}{HTML}{CC1914}
\definecolor{quarto-callout-warning-color}{HTML}{EB9113}
\definecolor{quarto-callout-tip-color}{HTML}{00A047}
\definecolor{quarto-callout-caution-color}{HTML}{FC5300}
\definecolor{quarto-callout-color-frame}{HTML}{acacac}
\definecolor{quarto-callout-note-color-frame}{HTML}{4582ec}
\definecolor{quarto-callout-important-color-frame}{HTML}{d9534f}
\definecolor{quarto-callout-warning-color-frame}{HTML}{f0ad4e}
\definecolor{quarto-callout-tip-color-frame}{HTML}{02b875}
\definecolor{quarto-callout-caution-color-frame}{HTML}{fd7e14}
\makeatother
\makeatletter
\@ifpackageloaded{bookmark}{}{\usepackage{bookmark}}
\makeatother
\makeatletter
\@ifpackageloaded{caption}{}{\usepackage{caption}}
\AtBeginDocument{%
\ifdefined\contentsname
  \renewcommand*\contentsname{Table of contents}
\else
  \newcommand\contentsname{Table of contents}
\fi
\ifdefined\listfigurename
  \renewcommand*\listfigurename{List of Figures}
\else
  \newcommand\listfigurename{List of Figures}
\fi
\ifdefined\listtablename
  \renewcommand*\listtablename{List of Tables}
\else
  \newcommand\listtablename{List of Tables}
\fi
\ifdefined\figurename
  \renewcommand*\figurename{Figure}
\else
  \newcommand\figurename{Figure}
\fi
\ifdefined\tablename
  \renewcommand*\tablename{Table}
\else
  \newcommand\tablename{Table}
\fi
}
\@ifpackageloaded{float}{}{\usepackage{float}}
\floatstyle{ruled}
\@ifundefined{c@chapter}{\newfloat{codelisting}{h}{lop}}{\newfloat{codelisting}{h}{lop}[chapter]}
\floatname{codelisting}{Listing}
\newcommand*\listoflistings{\listof{codelisting}{List of Listings}}
\makeatother
\makeatletter
\makeatother
\makeatletter
\@ifpackageloaded{caption}{}{\usepackage{caption}}
\@ifpackageloaded{subcaption}{}{\usepackage{subcaption}}
\makeatother
\makeatletter
\@ifpackageloaded{sidenotes}{}{\usepackage{sidenotes}}
\@ifpackageloaded{marginnote}{}{\usepackage{marginnote}}
\makeatother
\makeatletter
\@ifpackageloaded{qrcode}{}{\usepackage{qrcode}}
\makeatother

\ifLuaTeX
  \usepackage{selnolig}  % disable illegal ligatures
\fi
\usepackage{bookmark}

\IfFileExists{xurl.sty}{\usepackage{xurl}}{} % add URL line breaks if available
\urlstyle{same} % disable monospaced font for URLs
\hypersetup{
  pdftitle={Basic Statistics for the Social Sciences},
  pdfauthor={Konstantinos I. Bougioukas, PhD},
  colorlinks=true,
  linkcolor={blue},
  filecolor={Maroon},
  citecolor={Blue},
  urlcolor={Blue},
  pdfcreator={LaTeX via pandoc}}


\title{Basic Statistics for the Social Sciences}
\usepackage{etoolbox}
\makeatletter
\providecommand{\subtitle}[1]{% add subtitle to \maketitle
  \apptocmd{\@title}{\par {\large #1 \par}}{}{}
}
\makeatother
\subtitle{1st Edition}
\author{Konstantinos I. Bougioukas, PhD}
\date{2025-12-24}

\begin{document}
\maketitle

\renewcommand*\contentsname{Table of contents}
{
\hypersetup{linkcolor=}
\setcounter{tocdepth}{2}
\tableofcontents
}

\bookmarksetup{startatroot}

\chapter*{Welcome!}\label{welcome}
\addcontentsline{toc}{chapter}{Welcome!}

\markboth{Welcome!}{Welcome!}

This open course introduces students to key concepts and statistical
methods used in the quantitative social and behavioral sciences to
describe and test hypotheses about the social world and human behavior.
Students will learn to:

\begin{itemize}
\item
  Describe and summarize data distributions.
\item
  Formulate and test various types of research hypotheses.
\item
  Analyze associations between factors, characteristics, or events.
\item
  Interpret and critically evaluate published statistics.
\end{itemize}

The student may also learn about the proper use of graphing and
statistical software.

\qrcode[]{https://statsocial.netlify.app/}
    

\part{Statistical Thinking}

\chapter{Introduction}\label{sec-introduction}

When we have finished this chapter, we should be able to:

\begin{tcolorbox}[enhanced jigsaw, bottomrule=.15mm, title={\includegraphics[width=1em,height=1em]{intro_files/figure-pdf/fa-icon-fd658532f1071e4d71f26381511f9b57.pdf}
Learning objectives}, toprule=.15mm, opacitybacktitle=0.6, coltitle=black, opacityback=0, rightrule=.15mm, leftrule=.75mm, colback=white, bottomtitle=1mm, toptitle=1mm, colframe=quarto-callout-caution-color-frame, arc=.35mm, breakable, titlerule=0mm, left=2mm, colbacktitle=quarto-callout-caution-color!10!white]

\begin{itemize}
\tightlist
\item
  Distinguish descriptive from inferential statistics.
\item
  Understand the importance of data and identify the main sources of
  primary data.
\item
  Comprehend the importance of social statistics.
\item
  Understand the difference between dependent and independent variables,
  and be introduced to the concept of confounding.
\item
  Explain the difference between qualitative and quantitative data, and
  identify the type of any given variable.
\end{itemize}

\end{tcolorbox}

\section{Why learn basic statistics?}\label{why-learn-basic-statistics}

Knowledge of the fundamental principles of statistics, data analysis,
and research methodology is essential for professionals in psychology
and other social sciences today for three main reasons:

\begin{itemize}
\item
  \textbf{Understanding key statistical indicators:} This knowledge
  allows professionals to interpret data and indicators from sources
  such as ELSTAT, EUROSTAT, and WHO. With this expertise, they can
  effectively assess and verify relevant information from around the
  world, hereby helping to counter misinformation. Key examples of these
  indicators include population health status, social inequalities,
  rates of violence and crime, unemployment rates, educational
  attainment levels, poverty rates, and access to healthcare services.
\item
  \textbf{Critically analyzing research studies:} Professionals are
  better equipped to evaluate research studies and their findings,
  ensuring they can identify strengths, weaknesses, and potential
  biases.
\item
  \textbf{Conducting independent research:} Familiarity with statistics
  enables professionals who conduct their own research to collaborate
  more effectively with statistical experts, enhancing the overall
  quality of their work.
\end{itemize}

\section{The discipline of
Statistics}\label{the-discipline-of-statistics}

The word ``statistics'' originates from the Latin \textbf{``status''}.
Initially, it referred to the political state of a region, with
``statista'' used for recording information like censuses or data on a
state's wealth. Over time, the meaning and use of statistics broadened,
and its scope evolved.

Today, \textbf{Statistics} is an applied mathematical science that,
according to Croxton and Cowden, can be defined as ``the science of
collection, presentation, analysis, and interpretation of
\textbf{numerical data}''.

Statistics includes different theoretical frameworks such as
\textbf{traditional} (frequentist) statistics and \textbf{Bayesian}
statistics. In this course, we will cover classical parametric and
nonparametric statistical tests of traditional statistics.

Relying heavily on \textbf{probability theory} and empirical methods,
statistics aims to describe and summarize data as well as to draw
inferences about the population from which the data are derived.
Therefore, the discipline of traditional (frequentist) statistics
includes two main branches (Figure~\ref{fig-branches}):

\begin{itemize}
\item
  \textbf{Descriptive} statistics that includes measures of frequency
  and measures of location and dispersion. It also includes a
  description of the shape of the data distributions.
\item
  \textbf{Inferential} statistics that aims at generalizing conclusions
  made on a sample to a whole population. It includes estimation and
  hypothesis testing.
\end{itemize}

\begin{figure}

\centering{

\includegraphics[width=9in,height=5.25in]{intro_files/figure-latex/mermaid-figure-1.png}

}

\caption{\label{fig-branches}The discipline of statistics and its two
branches, descriptive statistics and inferential statistics}

\end{figure}%

In a research study, \textbf{both} descriptive and inferential
statistics are commonly used. First, researchers present descriptive
statistics (e.g., demographic data, baseline characteristics) to provide
a clear snapshot of the sample. Then, inferential statistics are applied
to test hypotheses and draw conclusions about the broader population
from which the sample was drawn.

\section{Talking about data}\label{talking-about-data}

\subsection{The ``Age of data''}\label{the-age-of-data}

\includegraphics[width=2.08333in,height=\textheight]{images/gwas.png}\hfill

We are living in the ``Age of Data'', where every day, an astonishing
2.5 quintillion (\(10^{18}\)) bytes of data are generated worldwide. An
example of how this era is transforming scientific research can be seen
in large-scale multi-ancestry genome-wide association studies (GWAS),
which are used to uncover the genetic basis of complex conditions like
anxiety disorders. In a recent study, researchers analyzed genomic data
from more than 1.2 million individuals across diverse populations,
identifying more than 100 genes associated with stress and anxiety
(Friligkou et al. 2024).

A second example is chatbots, which are designed to simulate
conversations with users. ELIZA (1966) was one such program. Its famous
``DOCTOR'' script emulated a psychotherapist by rephrasing user
statements as questions (Figure~\ref{fig-eliza}). Modern chatbots, such
as ChatGPT, Gemini and Copilot, are large language models trained on
\textbf{vast amounts of text data} (e.g., books, articles, websites, and
user-generated content), hence the term ``large''. Using deep learning
techniques, these models aim to understand and generate human-like text,
effectively responding to a wide range of user queries.

\begin{figure}

\centering{

\includegraphics[width=0.75\textwidth,height=\textheight]{images/eliza.png}

}

\caption{\label{fig-eliza}The most famous scenario simulated a
psychotherapist of the Rogerian school.}

\end{figure}%

~

Next is a YouTube video that explores the history of Eliza and chatbots.

\url{https://www.youtube.com/watch?v=zhxNI7V2IxM}

\subsection{Structure of data}\label{structure-of-data}

There are three main data structures:

\begin{itemize}
\item
  \textbf{Structured data} generally refer to highly organized tabular
  data, facilitating straightforward search, analysis, and processing.
  Examples include data stored in spreadsheets, such as Excel files, or
  in comma-separated values (CSV) files.
\item
  \textbf{Semi-Structured data} are a form of structured data that do
  not follow a strict tabular format but still have some organizational
  properties. For example, emails are semi-structured; they include
  fields like sender, recipient, subject, date, and time, and are also
  organized into folders such as Inbox, Sent, and Trash.
\item
  \textbf{Unstructured data} refer to information that lacks a
  predefined format or organization, such as open text (e.g., social
  media posts), images, videos, and other forms of multimedia.
\end{itemize}

In this course, we use data organized in a structured format, such as
spreadsheets. Tabular data refer to data organized in a table with rows
and columns (Figure~\ref{fig-datable}). Each row represents an
observation (or record), corresponding to a statistical unit in the
dataset. The columns represent the variables (or characteristics) of
interest. Cells are the individual units where rows and columns
intersect. Each cell contains a data value corresponding to the
observation for that row and the variable for that column.

\begin{figure}

\centering{

\includegraphics[width=0.85\textwidth,height=\textheight]{images/datable.png}

}

\caption{\label{fig-datable}A typical excel spreadsheet with row and
columns.}

\end{figure}%

\subsection{Sources of social and health
data}\label{sources-of-social-and-health-data}

Data in the social and health sciences refers to the information
collected and analyzed to understand human behavior, societal structures
and interactions. This data originates from various sources, each
offering unique insights into different aspects of human and societal
dynamics. These sources include:

\begin{itemize}
\item
  \textbf{Self-Reports:}
  \includegraphics[width=0.36458in,height=0.34375in]{images/q_icon.png}
  These are collected through interviews, questionnaires, and surveys,
  capturing individual experiences, behaviors, and attitudes. They
  provide qualitative insights that enhance our understanding of
  personal perspectives and social phenomena.
\item
  \textbf{Internet and Social Media:}
  \includegraphics[width=0.36458in,height=0.34375in]{images/internet_icon.png}
  These platforms generate vast amounts of data on online interactions,
  behaviors, and social trends, offering valuable information on how
  people communicate and engage in the digital space.
\item
  \textbf{Wearable Technology:}
  \includegraphics[width=0.36458in,height=0.34375in]{images/wearable_icon.png}
  Devices such as smartwatches, fitness trackers, smart glasses, and
  smart clothing equipped with sensors have become revolutionary tools
  for tracking and monitoring physiological and behavioral metrics in
  real-time. They provide critical insights into health, fitness, and
  daily habits.
\item
  \textbf{Electronic Health Records (EHRs):}
  \includegraphics[width=0.36458in,height=0.34375in]{images/ehr_icon.png}
  EHRs offer detailed information about patients' medical histories,
  treatments, and health outcomes, facilitating research on health
  trends and the effectiveness of interventions.
\item
  \textbf{Health Surveillance Systems:}
  \includegraphics[width=0.36458in,height=0.34375in]{images/surveillance_icon.png}
  These systems continuously monitor and analyze trends in real-time,
  including disease outbreaks, vaccination uptake, public health
  patterns, substance abuse trends, and crime statistics, thereby
  informing timely interventions and policy decisions.
\item
  \textbf{Clinical Registries:}
  \includegraphics[width=0.36458in,height=0.34375in]{images/cl_registry_icon.png}
  These collect data on patients with specific medical conditions or
  treatments, providing valuable insights into health outcomes and
  social determinants of health.
\item
  \textbf{Biobanks:}
  \includegraphics[width=0.36458in,height=0.34375in]{images/biobank_icon.png}
  Biobanks store biological samples (e.g., blood, tissue) along with
  health and lifestyle data, enabling research into the intersections of
  genetics, environment, and social factors.
\end{itemize}

\subsection{From data to knowledge and
decisions}\label{from-data-to-knowledge-and-decisions}

Social, behavioral, and biomedical \textbf{data} can be transformed into
\textbf{information}. This information can evolve into
\textbf{knowledge} when social scientists and stakeholders interpret and
understand it, allowing them to make \textbf{informed decisions}, shape
policies, and implement interventions that address societal and
health-related challenges more effectively (Figure~\ref{fig-info}).

\begin{figure}

\centering{

\includegraphics[width=0.9\textwidth,height=\textheight]{images/info.png}

}

\caption{\label{fig-info}From data to knowledge and action.}

\end{figure}%

\section{Social Statistics}\label{social-statistics}

\textbf{Social statistics} is a field of statistics applied in the
social sciences to study \textbf{social phenomena} and \textbf{trends}.
It employs various statistical methods of data collection, such as
censuses, social surveys, and administrative records. These methods are
commonly used by international organizations, government agencies,
institutions, and researchers to analyze data related to social life,
human behavior, and society as a whole.

The primary goal of social statistics is to provide objective,
quantitative evidence that aids in understanding and interpreting social
issues, ultimately informing the development of appropriate social
policies.

Let's explore some examples of \textbf{official statistics} from
national governments and international organizations.

\textbf{Example 1}

\includegraphics[width=1.04167in,height=\textheight]{images/elstat_icon.png}\hfill

Through its quarterly report \emph{Greece in Figures}, the Hellenic
Statistical Authority
(\href{https://www.statistics.gr/en/greece-in-figures}{ELSTAT}) presents
current and detailed statistical insights into Greece's population,
social structure, and economy.

An important socio-economic indicator at country level is the percentage
of people at risk of poverty or social exclusion. Poverty and social
exclusion severely disrupt the lives of those affected, perpetuating a
cycle of disadvantage that can last for generations. This cycle not only
limits personal growth but also poses significant challenges to social
cohesion and economic stability within communities.

\begin{figure}

\centering{

\includegraphics[width=0.7\textwidth,height=\textheight]{images/poverty.png}

}

\caption{\label{fig-poverty}The percentage of people at risk of poverty
or social exclusion in Eurozone. Greece (EL) is highlihgted.}

\end{figure}%

In 2023, over a quarter (26.1\%) of the Greek population was at risk of
poverty or social exclusion, ranking Greece (EL) among the highest in
the Eurozone, just behind Spain (ES; 26.5\%), as shown in
Figure~\ref{fig-poverty}.

~

\textbf{Example 2}

\includegraphics[width=1.04167in,height=\textheight]{images/violence.png}\hfill

Gender inequalities and violence against women and girls are urgent
social and public health issues that require immediate attention. A
report, based on data from 2018 and published in May 2021, conducted by
the World Health Organization, the London School of Hygiene and Tropical
Medicine and the South African Medical Research Council, presents
findings on the prevalence of violence against women.

The report found that the global lifetime prevalence of intimate partner
violence among ever-partnered women was \textbf{30.0\%} (95\% CI =
27.8\% to 32.2\%).

It is important to note that this estimated percentage (30\%) is
accompanied by a confidence interval (CI), which indicates a range from
27.8\% to 32.2\%. This interval reflects the precision of the estimate,
and we will discuss it further in the subsequent sections of the course.

The report also mentioned that the prevalence of intimate partner
violence among ever-partnered women was highest in the WHO African,
Eastern Mediterranean and South-East Asia Regions with an estimated
prevalence of approximately 37\%. In contrast, the prevalence was lower
in the high-income regions of America (23\%) and the European and
Western Pacific Regions (25\%). This indicates that the prevalence of
violence against women can differ significantly among regions,
influenced by a range of socio-economic factors. In statistical terms,
this suggests that there is a large \textbf{variance} in the data.

\includegraphics[width=3.125in,height=\textheight]{images/kirabati.png}\hfill

The report also references the
\href{https://iris.who.int/handle/10665/207758}{Kiribati case}; a
country consisting of 33 islands situated in the central Pacific Ocean.
The
\href{https://pacific.unfpa.org/sites/default/files/pub-pdf/Kiribati-Family-Health-and-Support-Study_0.pdf}{Kiribati
family health study} revealed an alarming prevalence of violence against
women in Kiribati, with 68\% of women aged 15--49 who had been in a
relationship reporting experiences of violence (emotional, physical,
and/or sexual) from a partner. This finding highlights how social
statistics can assist in identifying \textbf{rare cases} (extreme
values) to raise global concern and prompt immediate action.

The following YouTube video from \emph{UN Women} is a campaign aimed at
raising awareness about violence against women and girls.

\url{https://www.youtube.com/watch?v=W_ZPHPutN-c}

~

\section{Variables}\label{variables}

\subsection{Independent and dependent
variables}\label{independent-and-dependent-variables}

A \textbf{variable} is a quantity or characteristic that is free to
vary, or take on different values. Researchers design studies to test
whether changes to one or more variables are associated with changes in
another variable of interest.

\includegraphics[width=2.08333in,height=\textheight]{images/falls.png}\hfill

For example, if researchers hypothesize that a psychological
intervention can help prevent falls in older adults living in the
community more effectively than a standard approach, they could create a
study to test this hypothesis. Participants would be randomly assigned
to one of two groups: the experimental group, which receives the
psychological intervention aimed at preventing falls, and the control
group, which receives the usual care.

In this example, the type of intervention each participant received
(i.e., the psychological intervention vs.~usual care) is the
\textbf{independent variable} (X), as it is the variable that the
researchers manipulate. The \textbf{dependent} variable (Y), or the
outcome variable, is the rate of falls over a time period, as it
reflects the effect or outcome that the researchers measure to determine
whether the intervention has an impact on reducing falls among older
adults living in the community.

An \textbf{independent variable} (X) is the variable that is changed or
controlled in a research study to examine its effect on another variable
(Y), which represents the outcome being measured.

A \textbf{dependent variable} (Y) is the variable that is measured and
assessed in a research study, influenced by the independent variables
(Xs) being studied. It represents the outcome of the study.

It is important to note that we can construct both simple bivariate
models and more complex, realistic multivariable models:

\begin{itemize}
\tightlist
\item
  \textbf{Diagram of a Bivariate Model (X → Y)}
\end{itemize}

In this model, X is assumed to influence or predict Y. It's commonly
used to explore simple cause-and-effect associations or correlations,
such as how a psychological intervention (X) might affect rate of falls
in elderly (Y).

\includegraphics[width=4.91in,height=0.72in]{intro_files/figure-latex/mermaid-figure-5.png}

\begin{itemize}
\tightlist
\item
  \textbf{Diagram of Multivariable Model (X1, X2, X3 → Y)}
\end{itemize}

This model allows for a more complex analysis, where several factors are
considered simultaneously. For example, it can be used to study how a
psychological intervention, gender, and balance (X1, X2, X3) together
might affect rate of falls in elderly (Y), helping to capture the
combined effects of multiple influences on the outcome.

\includegraphics[width=4.91in,height=2.86in]{intro_files/figure-latex/mermaid-figure-4.png}

~

\subsection{Confounding variable}\label{confounding-variable}

It is often essential to analyze the association between an independent
variable (X) and a dependent variable (Y) while considering confounding
variables, which should be controlled to prevent distortion of results.

A confounding variable is defined as one that is related with both the
independent and the dependent variables and does not lie on the causal
pathway between them.

For example, consider a study investigating the link between caffeine
intake (X) and lung cancer (Y). Without accounting for confounding
variables, such as smoking, the results may suggest a misleading
association between caffeine and occurrence of lung cancer that is not
truly causal.

\begin{figure}

\centering{

\includegraphics[width=0.7\textwidth,height=\textheight]{images/confounder.png}

}

\caption{\label{fig-confounder}From data to knowledge and action.}

\end{figure}%

Smoking may be a potential confounding variable because it is:

\begin{itemize}
\item
  \textbf{related to caffeine consumption:} Smokers often consume more
  caffeine (e.g., through coffee or energy drinks).
\item
  \textbf{direct Cause of lung cancer:} Smoking is a well-established
  cause of lung cancer and significantly increases the risk of
  developing the disease.
\item
  \textbf{not directly part of the causal pathway:} Smoking does not lie
  in the causal pathway between caffeine and lung cancer.
\end{itemize}

\begin{tcolorbox}[enhanced jigsaw, bottomrule=.15mm, title={\includegraphics[width=1em,height=1em]{intro_files/figure-pdf/fa-icon-ee56651b74f8b687021aa43846e40a59.pdf}
Comment}, toprule=.15mm, opacitybacktitle=0.6, coltitle=black, opacityback=0, rightrule=.15mm, leftrule=.75mm, colback=white, bottomtitle=1mm, toptitle=1mm, colframe=quarto-callout-tip-color-frame, arc=.35mm, breakable, titlerule=0mm, left=2mm, colbacktitle=quarto-callout-tip-color!10!white]

The causal pathway refers to the sequence of events or mechanisms
through which an independent variable influences a dependent variable.
It demonstrates how changes in the independent variable lead to changes
in the dependent variable, often involving intermediary variables.

\includegraphics[width=6.97in,height=0.72in]{intro_files/figure-latex/mermaid-figure-3.png}

\end{tcolorbox}

Confounding can be controlled either before or after a study is
conducted. Various methods are available to control for confounding,
including matching, stratification, and more advanced multivariate
techniques (Grimes and Schulz 2002).

\begin{itemize}
\item
  \textbf{Pairwise matching:} In a case-control study where smoking is
  considered a confounding variable, cases and controls can be matched
  based on smoking status.
\item
  \textbf{Stratification:} After a study is conducted, results can be
  stratified by levels of the confounding variable. In the smoking
  example, results would be calculated separately for smokers and
  non-smokers to determine if the same effect occurs independently of
  smoking.
\item
  \textbf{Multivariate techniques:} Mathematical modelling examines the
  potential effect of one variable while simultaneously controlling for
  the effect of many other variables.
\end{itemize}

\section{Types of data in variables}\label{types-of-data-in-variables}

Data in variables can be either \textbf{categorical} or
\textbf{numerical} (otherwise known as qualitative and quantitative) in
nature (Figure~\ref{fig-data_types}):

\begin{figure}

\centering{

\includegraphics[width=9in,height=3.97in]{intro_files/figure-latex/mermaid-figure-2.png}

}

\caption{\label{fig-data_types}Broad classification of the different
types of data with examples.}

\end{figure}%

~

\begin{tcolorbox}[enhanced jigsaw, bottomrule=.15mm, title={Note}, toprule=.15mm, opacitybacktitle=0.6, coltitle=black, opacityback=0, rightrule=.15mm, leftrule=.75mm, colback=white, bottomtitle=1mm, toptitle=1mm, colframe=quarto-callout-tip-color-frame, arc=.35mm, breakable, titlerule=0mm, left=2mm, colbacktitle=quarto-callout-tip-color!10!white]

The \textbf{type of data} in variables is an important factor in
determining the most appropriate \textbf{statistical analysis} of the
data.

\end{tcolorbox}

\subsection{Categorical data}\label{categorical-data}

\textbf{A. Nominal data}

Nominal data consists of distinct, unordered categories that are labeled
but not measured, only \textbf{counted}. These categories can be binary,
such as diagnosed/not diagnosed with depression, or they can have more
than two categories, such as \textbf{eye color} (e.g., brown, blue,
green, gray) or \textbf{type of therapy} (e.g., cognitive-behavioral
therapy, psychoanalysis, humanistic therapy).

\begin{tcolorbox}[enhanced jigsaw, bottomrule=.15mm, title=\textcolor{quarto-callout-important-color}{\faExclamation}\hspace{0.5em}{Numerical representation of categories are just codes}, toprule=.15mm, opacitybacktitle=0.6, coltitle=black, opacityback=0, rightrule=.15mm, leftrule=.75mm, colback=white, bottomtitle=1mm, toptitle=1mm, colframe=quarto-callout-important-color-frame, arc=.35mm, breakable, titlerule=0mm, left=2mm, colbacktitle=quarto-callout-important-color!10!white]

We can represent diagnosed/not diagnosed with depression as 1/0 and
cognitive-behavioral therapy/psychoanalysis/humanistic therapy as 1/2/3
for therapy type. Unlike numerical data, the numbers assigned to
categories do not have mathematical meaning; they are merely
\textbf{codes}.

\end{tcolorbox}

~

\textbf{B. Ordinal data}

When categories can be ordered, the data are of \textbf{ordinal type}.
For example, patients may rate their \textbf{pain} as minimal, moderate,
severe, or unbearable. In this case, there is a natural order to the
values, as moderate pain is more intense than minimal but less than
severe. Another common example of ordinal data is the \textbf{Likert
scale}, where respondents might indicate their level of agreement with a
statement on a scale such as from 1 (strongly disagree) to 5 (strongly
agree).

\begin{tcolorbox}[enhanced jigsaw, bottomrule=.15mm, title=\textcolor{quarto-callout-important-color}{\faExclamation}\hspace{0.5em}{IMPORTANT}, toprule=.15mm, opacitybacktitle=0.6, coltitle=black, opacityback=0, rightrule=.15mm, leftrule=.75mm, colback=white, bottomtitle=1mm, toptitle=1mm, colframe=quarto-callout-important-color-frame, arc=.35mm, breakable, titlerule=0mm, left=2mm, colbacktitle=quarto-callout-important-color!10!white]

\begin{itemize}
\item
  Ordinal data are often transformed into binary data to simplify
  analysis, presentation, and interpretation, though this can result in
  a \textbf{loss of information}.
\item
  Although the \textbf{Likert scale} consists of ordered categories, the
  levels can be \textbf{treated like numeric data}, allowing scores from
  a questionnaire to be summed or averaged for a comprehensive
  assessment of respondents' attitudes or feelings (total score of the
  participant).
\end{itemize}

\end{tcolorbox}

\subsection{Numerical data}\label{numerical-data}

\textbf{A. Discrete data}

Discrete data can take only a \textbf{finite number of values} (usually
integers) in a range, for example, the number of children in a family or
the number of days missed from work. Other examples are often
\textbf{counts per unit of time} such as the number of visits to the
psychotherapist in a year, or the number of epileptic seizures a patient
has per month.

In practice discrete data are often treated in statistical analyses as
if they were ordinal data. Although this approach may be acceptable, it
may not fully optimize the use of our data.

\vspace{15pt}

\textbf{B. Continuous data}

Continuous data are numbers (usually with units) that can
\textbf{theoretically} take \textbf{any value} within a given range.
Examples of continuous variables include \textbf{height},
\textbf{weight}, \textbf{body temperature}, and \textbf{reaction time}.
However, in practice, these variables are often measured
\textbf{discretely}, constrained by the precision of measuring
instruments and the study's specific objectives. For instance, although
height can be a continuous value, it is typically recorded in discrete
steps. A person may measure 172.345 cm tall, but this would usually be
recorded as 172 cm.

\begin{tcolorbox}[enhanced jigsaw, bottomrule=.15mm, title=\textcolor{quarto-callout-important-color}{\faExclamation}\hspace{0.5em}{Categorization of numerical data leads to a loss of information}, toprule=.15mm, opacitybacktitle=0.6, coltitle=black, opacityback=0, rightrule=.15mm, leftrule=.75mm, colback=white, bottomtitle=1mm, toptitle=1mm, colframe=quarto-callout-important-color-frame, arc=.35mm, breakable, titlerule=0mm, left=2mm, colbacktitle=quarto-callout-important-color!10!white]

It is important to note that continuous data are often categorized to
create categorical variables. For example, body mass index (BMI)---a
continuous variable that measures weight relative to height---is
typically converted into an ordinal variable with four categories:
underweight, normal weight, overweight, and obese. However, dividing
continuous variables into categories results in a \textbf{considerable
loss of information}. Furthermore, the reverse transformation is
impossible; a categorical variable cannot be transformed into a
continuous variable.

\end{tcolorbox}

Next is a YouTube video by Associate Professor Mike Marin from the
University of British Columbia that comprehensively explains the
different types of variables.

\url{https://www.youtube.com/watch?v=ZxV-kf0yBss&t=2s}

\chapter{Descriptive statistics}\label{sec-descriptive}

When we have finished this chapter, we should be able to:

\begin{tcolorbox}[enhanced jigsaw, bottomrule=.15mm, title={\includegraphics[width=1em,height=1em]{descriptive_files/figure-pdf/fa-icon-fd658532f1071e4d71f26381511f9b57.pdf}
Learning objectives}, toprule=.15mm, opacitybacktitle=0.6, coltitle=black, opacityback=0, rightrule=.15mm, leftrule=.75mm, colback=white, bottomtitle=1mm, toptitle=1mm, colframe=quarto-callout-caution-color-frame, arc=.35mm, breakable, titlerule=0mm, left=2mm, colbacktitle=quarto-callout-caution-color!10!white]

\begin{itemize}
\tightlist
\item
  Use frequency counts and percentages to describe categorical
  variables.
\item
  Display frequency distributions for categorical variables using bar
  plots.
\item
  Use statistical measures such as mean, median, standard deviation and
  interquartile range to describe numerical variables.
\item
  Generate and assess visualizations like histograms, density plots, and
  box plots to understand the distribution and spread of numerical data.
\end{itemize}

\end{tcolorbox}

~

\section{Data}\label{data}

We will explore a dataset containing 258 participants (rows) and 8
variables (columns). The variables include sex (female/male), age (in
years), time spent on the internet and social media (in hours), the
total score (0-30) from responses to 10 questions on the Rosenberg
Self-Esteem Scale
(\href{https://socy.umd.edu/about-us/using-rosenberg-self-esteem-scale}{RSES}),
and the categorization of that score into three levels of self-esteem:
low (0-15), medium (16-19), and high (20-30) (García et al. 2019).

~

\includegraphics{descriptive_files/figure-pdf/unnamed-chunk-2-1.pdf}

~

\section{Summarizing categorical data (Frequency
Statistics)}\label{summarizing-categorical-data-frequency-statistics}

\subsection{One variable frequency tables and
plots}\label{one-variable-frequency-tables-and-plots}

The first step in analyzing a categorical variable is to count the
occurrences of each label and calculate their frequencies. This
collection of frequencies for all possible categories is known as the
\textbf{frequency distribution} of the variable. Additionally, we can
express these frequencies as proportions of the total number of
observations, which are referred to as \textbf{relative frequencies}. If
we multiply these proportions by 100, we obtain \textbf{percentages}
(\%).

~

\textbf{Sex variable}

Let's create a a frequency table for the \texttt{sex} variable:

\begin{figure}

\centering{

\includegraphics[width=0.7\textwidth,height=\textheight]{images/tb001.png}

}

\caption{\label{fig-tb001}Fequency table for the sex.}

\end{figure}%

The table displays the following:

\begin{itemize}
\item
  \textbf{Absolute frequency (n):} The number of participants in each
  category (male: 109, female: 149).
\item
  \textbf{Percentage (\%):} The proportion of participants in each
  category relative to the total number of participants (relative
  frequency) multiplied by 100\% (male: 109/258 x 100 = 42.2\%, female:
  149/258 x 100 = 57.8\%). Note that the percentages sum up to 100\%
  (42.2\% + 57.8\%).
\item
  \textbf{Cumulative percentage (\%):} The sum of the percentage
  contributions of all categories up to and including the current one.
  For example, for the male category, the cumulative percentage is
  42.2\%. When combining male and female categories, the cumulative
  percentage is 42.2\% + 57.8\% = 100\%. Therefore, the final cumulative
  percentage must equal 100\%.
\end{itemize}

While frequency tables are extremely useful, plotting the data often
provides a clearer presentation. For categorical variables, such as sex,
it is straightforward to display the number of occurrences in each
category using \textbf{bar plots}. The x-axis typically represents the
categories of the variable---in this case, ``male'' and ``female''. The
y-axis represents the frequency or count of occurrences for each
category.

\begin{figure}

\centering{

\includegraphics{descriptive_files/figure-pdf/fig-barsex1-1.pdf}

}

\caption{\label{fig-barsex1}Bar plot showing the frequency distribution
of the sex.}

\end{figure}%

\vspace{15pt}

If the y-axis represents percentages (\%), then each bar's height
corresponds to the percentage of participants in that category. For
example, the percentage of female participants is 57.8\%.

\begin{figure}

\centering{

\includegraphics{descriptive_files/figure-pdf/fig-barsex2-1.pdf}

}

\caption{\label{fig-barsex2}Bar plot showing the percentage of
participants for each level of self-esteem.}

\end{figure}%

~

\textbf{Score\_cat variable (self-esteem)}

Similarly, we can create the frequency table for the \texttt{Score\_cat}
(self-esteem) variable:

\begin{figure}

\centering{

\includegraphics[width=0.7\textwidth,height=\textheight]{images/tb002.png}

}

\caption{\label{fig-tb002}Frequency table for the levels of
self-esteem.}

\end{figure}%

In the above table, we observe that 40.3\% (104 out of 258) of
participants have a low level of self-esteem. When we combine the low
and medium categories, the cumulative percentage is: 40.3\% + 32.6\% =
72.9\%. Finally, for all categories (low, medium, high), the cumulative
percentage sums to 72.9\% + 27.1\% = 100\%.

Figure~\ref{fig-baresteem1} illustrates the frequency distribution of
self-esteem. The horizontal axis (\emph{x-axis}) displays the different
self-esteem categories, ordered according to increasing self-esteem
levels, while the vertical axis (y-axis) shows the frequency of each
category.

\begin{figure}

\centering{

\includegraphics{descriptive_files/figure-pdf/fig-baresteem1-1.pdf}

}

\caption{\label{fig-baresteem1}Bar plot showing the frequency for each
level of self-esteem.}

\end{figure}%

Figure~\ref{fig-baresteem2} illustrates the distribution of self-esteem
using percentages. Most participants fall into the category of low
self-esteem, accounting for 40.3\% (172 out of 428), highlighting a
significant portion of the sample that may benefit from targeted
interventions or support.

\begin{figure}

\centering{

\includegraphics{descriptive_files/figure-pdf/fig-baresteem2-1.pdf}

}

\caption{\label{fig-baresteem2}Bar plot showing the percentage of
participants for each level of self-esteem.}

\end{figure}%

~

\begin{tcolorbox}[enhanced jigsaw, bottomrule=.15mm, title=\textcolor{quarto-callout-important-color}{\faExclamation}\hspace{0.5em}{Tips for simple bar plots}, toprule=.15mm, opacitybacktitle=0.6, coltitle=black, opacityback=0, rightrule=.15mm, leftrule=.75mm, colback=white, bottomtitle=1mm, toptitle=1mm, colframe=quarto-callout-important-color-frame, arc=.35mm, breakable, titlerule=0mm, left=2mm, colbacktitle=quarto-callout-important-color!10!white]

\begin{itemize}
\tightlist
\item
  All bars should have equal width and equal spacing between them.
\item
  The height of each bar should correspond to the data it represents.
\item
  The bars should be plotted against a \textbf{common zero-valued
  baseline}.
\end{itemize}

\end{tcolorbox}

\subsection{Two variable tables (Contingency tables) and
plots}\label{two-variable-tables-contingency-tables-and-plots}

\textbf{A. Frequency contingency table}

In addition to tabulating each variable separately, we might also be
interested in exploring the association between two categorical
variables. In this case, the resulting frequency table is a
cross-tabulation, where each combination of levels from both variables
is displayed. This type of table is called a \textbf{contingency table}
because it shows the frequency of each category in one variable (e.g.,
sex), contingent upon the specific levels of the other variable (e.g.,
self-esteem), as shown in Figure~\ref{fig-tb0}.

\begin{figure}

\centering{

\includegraphics[width=0.7\textwidth,height=\textheight]{images/tb0.png}

}

\caption{\label{fig-tb0}Frequency contingency table which cross
tabulates sex and self-esteem levels. Each cell value in the table
represents the count corresponding to the combination of categories from
the two variables.}

\end{figure}%

\textbf{Note:} The table also typically includes \textbf{row and column
totals}, also known as marginal totals, that sum the counts for each row
and column, respectively.

~

\textbf{B. Joint distribution contingency table}

A \textbf{joint distribution} contingency table displays both the
frequency of observations across categories of two variables and the
\textbf{percentage distributions} of those frequencies. The percentages
are calculated by dividing the frequency in each cell by the
\textbf{overall total} (258), then multiplying the result by 100. This
shows the percentage of the total observations that fall into each
category combination.

\begin{figure}

\centering{

\includegraphics[width=0.75\textwidth,height=\textheight]{images/tb1.png}

}

\caption{\label{fig-tb1}Joint distribution contingency table. The
percentages are calculated by dividing the frequency in each cell by the
overall total.}

\end{figure}%

~

\textbf{C. Conditional distribution contingency table}

Suppose we are interested in the distribution of \textbf{self-esteem
levels} within each sex group, meaning we are observing how self-esteem
vary \textbf{among males} and \textbf{among females}. By
\textbf{conditioning} on sex, we divide each cell's frequency by the
corresponding \textbf{row total} (row marginal total), rather than the
overall total. This method allows us to examine the \textbf{conditional
distribution} of self-esteem within each sex group. For example, the
percentage of participants with low self-esteem, given that the
participant is female, is calculated as (78/149) x 100 ≈ 52.4\%.

\begin{figure}

\centering{

\includegraphics[width=0.75\textwidth,height=\textheight]{images/tb2.png}

}

\caption{\label{fig-tb2}Conditional distribution contingency table. The
percentages are calculated by dividing the frequency in each cell by the
row marginal total.}

\end{figure}%

This data analysis indicates notable differences in self-esteem levels
between male and female participants. Specifically, the percentage of
female participants with low self-esteem (52.4\%) is substantially
greater than that of male participants (23.9\%).

\begin{tcolorbox}[enhanced jigsaw, bottomrule=.15mm, title={\includegraphics[width=1em,height=1em]{descriptive_files/figure-pdf/fa-icon-ee56651b74f8b687021aa43846e40a59.pdf}
Comment}, toprule=.15mm, opacitybacktitle=0.6, coltitle=black, opacityback=0, rightrule=.15mm, leftrule=.75mm, colback=white, bottomtitle=1mm, toptitle=1mm, colframe=quarto-callout-tip-color-frame, arc=.35mm, breakable, titlerule=0mm, left=2mm, colbacktitle=quarto-callout-tip-color!10!white]

If we are interested in the distribution of \textbf{sex} within each
self-esteem level, focusing on how the proportion of males and females
varies within each self-esteem category, we condition on self-esteem.
This means we divide each cell's frequency by the corresponding
\textbf{column total}. For example, among those with low self-esteem,
the percentage of females is (78/104) x 100 ≈ 75.0\%

\end{tcolorbox}

~

We can also graphically present the data in the table shown in
Figure~\ref{fig-tb2}. A side-by-side bar plot (Figure~\ref{fig-sidebar})
can facilitate easier visual comparisons.

\begin{figure}

\centering{

\includegraphics{descriptive_files/figure-pdf/fig-sidebar-1.pdf}

}

\caption{\label{fig-sidebar}Side-by-side bar plot showing by self-esteem
level and sex.}

\end{figure}%

Alternatively, we can create a stacked bar plot, where the bars are
segmented by self-esteem levels. Figure~\ref{fig-stackbar} illustrates a
\textbf{100\% stacked bar plot}, which displays the percentage of each
self-esteem level (low, medium, high) among male and female
participants, emphasizing the relative differences within each group.
For example, the plot shows that a higher proportion of females have low
self-esteem (52.4\%) compared to males (23.9\%), while males have a
higher proportion of medium (43.1\%) and high self-esteem (33\%)
compared to females.

\begin{figure}

\centering{

\includegraphics{descriptive_files/figure-pdf/fig-stackbar-1.pdf}

}

\caption{\label{fig-stackbar}A horizontal stacked bar plot showing the
distribution of self-esteem stratified by sex.}

\end{figure}%

\vspace{10pt}

\begin{tcolorbox}[enhanced jigsaw, bottomrule=.15mm, title=\textcolor{quarto-callout-warning-color}{\faExclamationTriangle}\hspace{0.5em}{Caution}, toprule=.15mm, opacitybacktitle=0.6, coltitle=black, opacityback=0, rightrule=.15mm, leftrule=.75mm, colback=white, bottomtitle=1mm, toptitle=1mm, colframe=quarto-callout-warning-color-frame, arc=.35mm, breakable, titlerule=0mm, left=2mm, colbacktitle=quarto-callout-warning-color!10!white]

One consideration when using stacked bar plots is the number of variable
levels: with many categories, stacked bar plots can become confusing.

\end{tcolorbox}

~

\section{Summarizing numerical data (Summary
Statistics)}\label{summarizing-numerical-data-summary-statistics}

Summary measures are \textbf{single numerical values} that summarize a
set of data. Numeric data can be described using two main types of
summary measures (@tbl-measures).

\begin{enumerate}
\def\labelenumi{\arabic{enumi}.}
\item
  Measures of \textbf{central location}: These describe the ``center''
  of the data distribution. Common examples include the mean, median,
  and mode.
\item
  Measures of \textbf{dispersion}: These quantify the spread of values
  around the central value. Examples include the range, interquartile
  range (IQR), variance, and standard deviation.
\end{enumerate}

~

\begin{longtable}[]{@{}
  >{\raggedright\arraybackslash}p{(\columnwidth - 2\tabcolsep) * \real{0.4167}}
  >{\raggedright\arraybackslash}p{(\columnwidth - 2\tabcolsep) * \real{0.5833}}@{}}
\caption{Common summary measures of central location and
dispersion}\label{tbl-measures}\tabularnewline
\toprule\noalign{}
\begin{minipage}[b]{\linewidth}\raggedright
\textbf{Measures of Central Location}
\end{minipage} & \begin{minipage}[b]{\linewidth}\raggedright
\textbf{Measures of Dispersion}
\end{minipage} \\
\midrule\noalign{}
\endfirsthead
\toprule\noalign{}
\begin{minipage}[b]{\linewidth}\raggedright
\textbf{Measures of Central Location}
\end{minipage} & \begin{minipage}[b]{\linewidth}\raggedright
\textbf{Measures of Dispersion}
\end{minipage} \\
\midrule\noalign{}
\endhead
\bottomrule\noalign{}
\endlastfoot
• Mean & • Variance \\
• Median & • Standard Deviation \\
• Mode & • Range (Minimum, Maximum) \\
& • Interquartile Range (1st and 3rd Quartiles) \\
\end{longtable}

~

Additionally, \textbf{measures of shape} such as the sample coefficients
of \textbf{skewness} and \textbf{kurtosis} provide further insights by
revealing the overall shape and characteristics of the distribution.

\subsection{Measures of central
location}\label{measures-of-central-location}

\textbf{A. Sample Mean or Average}

The arithmetic mean, or average, denoted as \(\bar{x}\), is calculated
by dividing the sum of all values in a set by the total number of values
in the set.

\begin{equation}\phantomsection\label{eq-mean}{\bar{x}= \frac{Sum \ of \ values}{Number \ of \ values}}\end{equation}

\textbf{\emph{Example}}

We subset the data to a smaller sample to make it easier to manually
calculate the summary measures. As an example, we will use the number of
hours per day spent on the internet by 15-year-old females with low
self-esteem, which are as follows:

\[ 5 \ \ \ \ \  4  \ \ \ \ \ 12 \ \ \ \ \  8 \ \ \ \ \  4 \]

To calculate the mean, we use the Equation~\ref{eq-mean}:

\[\bar{x} = \frac{5 + 4 + 12 + 8 + 4}{5} = \frac{33}{5} = 6.6 \ hours\]

~

To highlight how outliers affect the mean, suppose we add a value of 24
to the dataset. This new data point is considered an \textbf{outlier},
as it is substantially higher than the other values in the dataset.
Since 24 hours is the maximum possible in a single day, this extreme
value clearly stands out from the rest of the observations.

Our new dataset becomes:

\[ 5 \ \ \ \ \  4  \ \ \ \ \ 12 \ \ \ \ \  8 \ \ \ \ \  4 \ \ \ \ \  \underline{24}\]

We can determine the new mean, \(\bar{x}_{new}\), by adding up all six
values and dividing by six:

\[\bar{x}_{new}= \frac{5 + 4 + 12 + 8 + 4 + 24}{6} = \frac{57}{6} = 9.5 \ hours\]

After adding this outlier, the mean increased from 6.6 to 9.5. This
significant rise of 2.9 hours illustrates how outliers can distort the
average, making it less representative of the dataset.

~

\textbf{Advantages of mean}

\begin{itemize}
\tightlist
\item
  It uses all the data values in the calculation and is the balance
  point of the data.
\item
  It is algebraically defined and thus mathematically manageable.
\end{itemize}

\textbf{Disadvantages of mean}

\begin{itemize}
\tightlist
\item
  It is highly influenced by the presence of \textbf{outliers}---values
  that are abnormally high or low---making it a non-resistant summary
  measure.
\item
  It cannot be easily determined by simply inspecting the data and is
  usually not equal to any of the individual values in the sample.
\end{itemize}

~

\textbf{B. Median of the sample}

The \textbf{sample median}, denoted as \emph{md}, is an alternative
measure of location that is less sensitive to outliers than mean.

The median is calculated by first sorting the observed values
(i.e.~arranging them in an ascending or descending order) and selecting
the middle one. If the number of observations is \textbf{odd}, the
median corresponds to the number in the middle of the sorted values. If
the number of observations is \textbf{even}, the median is the average
of the two middle numbers.

\textbf{\emph{Example}}

First, we sort the observed values from smallest to largest:

\textbf{Observed values:}
\(\ \ \ \ \ 5 \ \ \ \ \  4  \ \ \ \ \ 12 \ \ \ \ \  8 \ \ \ \ \  4\)

\textbf{Sorted values:}
\(\ \ \ \ \ \ \ \ \  4 \ \ \ \ \  4 \ \ \ \ \ \  5 \ \ \ \ \ \  8  \ \ \ \ \ 12\)

The number of observations is 5, which is an \textbf{odd} number;
therefore, the median corresponds to the value in the middle of the
sorted data:

\[ 4 \ \ \ \ \ 4 \ \ \ \ \ \textcolor{black}{\textbf{5}} \ \ \ \ \ 8 \ \ \ \ \ 12 \]

\[md = 5 \ hours\] ~

Now let's examine how the median responds to an outlier by adding the
value of 24 to the data. In this case, the number of observations
becomes 6, which is an \textbf{even} number. The new median,
\(md_{new}\), is the average of the two middle numbers, 5 and 8:

\[ 4 \ \ \ \ \ 4 \ \ \ \ \ \textcolor{black}{\textbf{5}} \ \ \ \ \ \textcolor{black}{\textbf{8}} \ \ \ \ \ 12 \ \ \ \ \ \underline{24}\]

\[md_{new} = \frac{5 + 8}{2} = \frac{13}{2} = 6.5 \ hours\]

We observe that the median is not strongly influenced by the addition of
the outlier, as it only increased from 5 to 6.5. This demonstrates that
the median is more resistant to outliers compared to the mean.

~

\textbf{Advantages of median}

\begin{itemize}
\tightlist
\item
  It is resistant to extreme values (outliers) compared to the mean.
\end{itemize}

\textbf{Disadvantages of median}

\begin{itemize}
\tightlist
\item
  It ignores the actual values of the data points, potentially losing
  some information about the data.
\end{itemize}

~

\textbf{C. Mode of the sample}

Another measure of location is the \textbf{mode} of the sample.

Mode represents the value that occurs \textbf{most frequently} in a set
of data values.

\textbf{\emph{Example}}

In our example, the value of 4 appears twice in the data:

\[ 5 \ \ \ \ \ \textcolor{black}{\textbf{4}} \ \ \ \ \ 12 \ \ \ \ \ 8 \ \ \ \ \ \textcolor{black}{\textbf{4}} \]

Therefore, the mode is:

\[ Mode = 4\]

It's important to note that some datasets may not have a mode if each
value occurs only once. For example, if we replace one of the fours with
a three:

\[ 5 \ \ \ \ \  \underline{3}  \ \ \ \ \ 12 \ \ \ \ \  8 \ \ \ \ \  4 \]
In this case, no value repeats, so the dataset has no mode.

~

However, if we replace the 12 with an 8, the dataset becomes:

\[ 5 \ \ \ \ \  \textcolor{red}{\textbf{4}}  \ \ \ \ \ \underline{\textcolor{green}{\textbf{8}}} \ \ \ \ \  \textcolor{green}{\textbf{8}} \ \ \ \ \  \textcolor{red}{\textbf{4}} \]

Here, both 4 and 8 appear twice, making the dataset bimodal, meaning it
has two modes.

\[ \textcolor{red}{Mode1 = 4}\] and

\[ \textcolor{green}{Mode2 = 8}\]

~

\begin{tcolorbox}[enhanced jigsaw, bottomrule=.15mm, title={\includegraphics[width=1em,height=1em]{descriptive_files/figure-pdf/fa-icon-ee56651b74f8b687021aa43846e40a59.pdf}
Comment}, toprule=.15mm, opacitybacktitle=0.6, coltitle=black, opacityback=0, rightrule=.15mm, leftrule=.75mm, colback=white, bottomtitle=1mm, toptitle=1mm, colframe=quarto-callout-tip-color-frame, arc=.35mm, breakable, titlerule=0mm, left=2mm, colbacktitle=quarto-callout-tip-color!10!white]

While the mode is less commonly used for numerical data, it can be a
useful measure of central tendency for categorical data, representing
the category with the highest frequency.

\end{tcolorbox}

~

\subsection{Measures of dispersion}\label{measures-of-dispersion}

\textbf{A. Range of the sample}

The \textbf{range} is the difference between the maximum and minimum
values in a dataset.

\begin{equation}\phantomsection\label{eq-range}{range = Max - Min}\end{equation}

The \textbf{minimum (Min)} value represents the lowest value observed in
a dataset, while the \textbf{maximum (Max)} value represents the highest
value. These values provide valuable insights into the range and
potential outliers within the dataset.

\textbf{\emph{Example}}

Let's determine the range for the sorted data in our example:

\[ \textcolor{black}{\textbf{4}} \ \ \ \ \  4 \ \ \ \ \ 5  \ \ \ \ \ 8 \ \ \ \ \ \textcolor{black}{\textbf{12}}\]

The minimum is Min = 4 hours, and the maximum is Max = 12 hours.
Therefore, according to Equation~\ref{eq-range}:

\[ range = 12 - 4 = 8 \ hours\]

~

Let's add the extreme value of 24 to the data.

\[ \textcolor{black}{\textbf{4}} \ \ \ \ \ 4 \ \ \ \ \ 5  \ \ \ \ \ 8 \ \ \ \ \ 12 \ \ \ \ \ \underline{\textcolor{black}{\textbf{24}}}\]

In this case, the range becomes:

\[range = 24 -4 = 20 \ hours\]

The main disadvantages of the range as a measure of dispersion are its
sensitivity to outliers and the fact that it uses only the extreme
values, ignoring all other data points.

~

\textbf{B. Inter-quartile range of the sample}

In the presence of outliers, the interquartile range (IQR) can provide a
more accurate measure of the spread of the majority of the data. Before
we define the interquartile range (IQR), let's first clarify some basic
concepts, specifically quantiles and quartiles.

A quantile indicates the value below which a certain proportion of the
data falls. The most commonly used quantiles are known as
\textbf{quartiles}:

\begin{itemize}
\item
  \(Q_1\) (lower quartile) represents the value at which 25\% of the
  data falls below it and 75\% falls above it.
\item
  \(Q_2\) (median) is the middle value when the data is arranged in
  ascending or descending order.
\item
  \(Q_3\) (upper quartile) represents the value at which 75\% of the
  data falls below it and 25\% falls above it.
\end{itemize}

Interquartile range is the difference between the third quartile (or
upper quartile) and the first quartile (or lower quartile) in an ordered
data set.

\begin{equation}\phantomsection\label{eq-iqr}{IQR = Q_3 - Q_1}\end{equation}

Therefore, the IQR focuses on the middle 50\% of the dataset.

\begin{figure}

\centering{

\includegraphics[width=0.5\textwidth,height=\textheight]{images/iqr.png}

}

\caption{\label{fig-iqr}Quartiles and inter-quartile range.}

\end{figure}%

It's important to note that different statistical software packages may
produce slightly different quartiles and interquartile ranges (IQRs) for
the same dataset, especially when there are only a few values present.
This discrepancy is due to the \textbf{numerous definitions} of sample
quantiles used in statistical software packages (Hyndman and Fan 1996).
While discussing these differences is beyond the scope of this
introductory course, we will focus on the results provided by Jamovi.

\textbf{\emph{Example}}

In our example, the first quartile is \(Q_1 = 4\) hours, and the third
quartile is \(Q_3 = 8\) hours.

\[ 4 \ \ \ \ \  \textcolor{black}{\textbf{4}} \ \ \ \ \ 5  \ \ \ \ \ \textcolor{black}{\textbf{8}} \ \ \ \ \ 12\]

Therefore, the inter-quartile range is:

\[IQR = Q_3 - Q1 = 8 - 4 = 4\]

~

After adding an extreme value such as 24, our example sorted dataset is
as follows:

\[ 4 \ \ \ \ \ 4 \ \ \ \ \ 5  \ \ \ \ \ 8 \ \ \ \ \ 12 \ \ \ \ \ \underline{24}\]

We would expect \(Q_1\) to have a value between 4 and 5, and \(Q_3\) to
fall between 8 and 12. JAMOVI provides \(Q_1 = 4.25\) hours, and
\(Q_3 =11\) hours. Therefore, the new inter-quartile range is:

\[IQR_{new} = Q_3 - Q1 = 11 - 4.25 = 6.75\]

As with the range, greater variability in the data typically leads to a
larger IQR. However, unlike the range, the \textbf{IQR is resistant to
outliers}, as it is not influenced by observations below the first
quartile or above the third quartile.

~

\textbf{C. Sample variance}

Sample variance, denoted as \(s^2\), is a measure of spread of the data
based on the \textbf{deviations} of the data values from the mean.
However, when we average these deviations, the sum always equals zero.
This occurs because the mean acts as a balance point where the total
positive and negative deviations cancel each other out. To resolve this
issue, we calculate the variance using \textbf{squared deviations},
which ensures that all values contribute positively to the measure of
spread.

Mathematically, the sample variance, \(s^2\), is calculated as the sum
of the squared deviations from the sample mean, divided by the number of
observations minus 1.

\begin{equation}\phantomsection\label{eq-var}{s^2 = \frac{\text{Sum of squared deviations}}{\text{Number of values} - 1}}\end{equation}

\textbf{\emph{Example}}

The original values are:

\[ 5 \ \ \ \ \  4  \ \ \ \ \ 12 \ \ \ \ \  8 \ \ \ \ \  4 \]

and we have calculated the mean, \(\bar{x} = 6.6\). The deviation
(difference) from the mean for the first value is calculated as 5 - 6.6
= -1.6, and for the second value, it is 4 - 6.6 = -2.6, and so on.

Thus, according to Equation~\ref{eq-var}:

\[
\begin{align*}
s^2 &= \frac{(5 - 6.6)^2 + (4 - 6.6)^2 + (12 - 6.6)^2 + (8 - 6.6)^2 + (4 - 6.6)^2}{5 - 1} \\
    &= \frac{(-1.6)^2 + (-2.6)^2 + (5.4)^2 + (1.4)^2 + (-2.6)^2}{4} \\
    &= \frac{2.56 + 6.76 + 29.16 + 1.96 + 6.76}{4} \\
    &= \frac{47.20}{4} \\
    &= 11.80 \ hours^2
\end{align*}
\]

~

Now let's examine how the variance responds to an outlier by adding the
value of 24 to the data. The new mean is \(\bar{x}_{new} = 9.5\), and
the variance is calculated as follows:

\[
\begin{align*}
s^2 &= \frac{(5 - 9.5)^2 + (4 - 9.5)^2 + (12 - 9.5)^2 + (8 - 9.5)^2 + (4 - 9.5)^2 + (24 - 9.5)^2}{6 - 1} \\
    &= \frac{(-4.5)^2 + (-5.5)^2 + (2.5)^2 + (-1.5)^2 + (-5.5)^2 + (14.5)^2}{5} \\
    &= \frac{20.25 + 30.25 + 6.25 + 2.25 + 30.25 + 210.25}{5} \\
    &= \frac{299.5}{5} \\
    &= 59.9 \ hours^2
\end{align*}
\]

The variance is sensitive to outliers because it is based on the
\textbf{squared} deviations from the mean. As a result, even a single
extreme value can significantly increase the variance, making it a less
reliable measure of spread when outliers are present.

The variance, being expressed in square units, \textbf{is not} the
preferred metric for describing the variability of data.

~

\textbf{D. Standard deviation of the sample}

Standard deviation is one of the most common measures of spread and is
particularly useful for assessing how far the data points are
distributed from the mean.

Standard deviation, denoted as \emph{s} or \emph{sd}, represents the
typical distance of observations from the mean. It is calculated as the
square root of the sample variance.

\begin{equation}\phantomsection\label{eq-sd}{s = \sqrt{s^2}}\end{equation}

\textbf{\emph{Example}}

The standard deviation is:

\[s = \sqrt{11.8} = 3.44 \ hours\]

and is expressed in the same units as the original data values.

~

With the addition of the value 25, the standard deviation becomes:

\[s = \sqrt{59.9} = 7.74 \ hours\]

The standard deviation uses all observations in a dataset for its
calculation and is expressed in the same units as the original data.
However, it is sensitive to outliers, which can substantially influence
its value.

~

\subsection{Plots for continuous
variables}\label{plots-for-continuous-variables}

When visualizing continuous data, several types of plots can be used to
understand the distribution, spread, and overall patterns in the data.

In the following examples, we will use the \textbf{complete} dataset
consisting of 258 observations.

~

\textbf{A. Frequency histogram}

The most common way to present the frequency distribution of numerical
data, especially when there are many observations, is through a
\textbf{histogram}. Histograms visualize the data distribution as
\textbf{a series of bars without gaps} between them (unless a particular
bin has zero frequency), in contrast to bar plots. Each bar typically
represents a range of numeric values known as a \textbf{bin} (or class),
with the height of the bar indicating the \textbf{frequency} of
observations (counts) within that particular bin. Below are the
frequency histograms for \texttt{age} and \texttt{time\_spent}:

\begin{figure}

\begin{minipage}{0.50\linewidth}

\centering{

\includegraphics{descriptive_files/figure-pdf/fig-histog1-1.pdf}

}

\subcaption{\label{fig-histog1-1}Histogram of age.}

\end{minipage}%
%
\begin{minipage}{0.50\linewidth}

\centering{

\includegraphics{descriptive_files/figure-pdf/fig-histog1-2.pdf}

}

\subcaption{\label{fig-histog1-2}Histogram of time spent.}

\end{minipage}%

\caption{\label{fig-histog1}Histograms of age and time spent variables.}

\end{figure}%

In Figure~\ref{fig-histog1}(a), the \texttt{age} distribution exhibits a
symmetrical bell-shaped form, with the highest frequency occurring
around 18 years old. The participants' ages range approximately from 14
to 22 years. In Figure~\ref{fig-histog1}(b), the \texttt{time\_spent}
variable follows a right-skewed distribution, with a higher frequency
occurring around 3 hours and a range roughly from 0 to 18 hours.

\begin{tcolorbox}[enhanced jigsaw, bottomrule=.15mm, title={\includegraphics[width=1em,height=1em]{descriptive_files/figure-pdf/fa-icon-ee56651b74f8b687021aa43846e40a59.pdf}
Comment}, toprule=.15mm, opacitybacktitle=0.6, coltitle=black, opacityback=0, rightrule=.15mm, leftrule=.75mm, colback=white, bottomtitle=1mm, toptitle=1mm, colframe=quarto-callout-tip-color-frame, arc=.35mm, breakable, titlerule=0mm, left=2mm, colbacktitle=quarto-callout-tip-color!10!white]

The visual appearance of a histogram is greatly influenced by the choice
of the \textbf{binwidth} (the difference between the lower and upper
limits of the bin). If the binwidth is too small (resulting in a large
number of bins), the histogram may appear overly detailed and noisy,
making it difficult to discern meaningful patterns. Conversely, if the
binwidth is too large (resulting in a small number of bins), important
features of the data distribution may be obscured. Experimenting with
different binwidths helps us find the optimal setting that best
represents the data, resulting in a more informative visualization.

\end{tcolorbox}

~

To summarize, a histogram provides information on:

\begin{itemize}
\item
  The distribution of the data, whether it's symmetrical or
  asymmetrical, and the presence of any outliers.
\item
  The location of the peak(s) in the distribution.
\item
  The degree of variability within the data, indicating the spread and
  range covered by the data.
\end{itemize}

~

\textbf{B. Density plot}

A \textbf{density plot} is another way to represent the distribution of
numerical data, often seen as a smoother version of a histogram
(Figure~\ref{fig-density1}). Moreover, density curves are typically
scaled so that the area under the curve equals one.

~

\begin{figure}

\begin{minipage}{0.50\linewidth}

\centering{

\includegraphics{descriptive_files/figure-pdf/fig-density1-1.pdf}

}

\subcaption{\label{fig-density1-1}Histogram of age.}

\end{minipage}%
%
\begin{minipage}{0.50\linewidth}

\centering{

\includegraphics{descriptive_files/figure-pdf/fig-density1-2.pdf}

}

\subcaption{\label{fig-density1-2}Histogram of time spent.}

\end{minipage}%

\caption{\label{fig-density1}Density plots of age and time spent
variables.}

\end{figure}%

~

\textbf{C. Box Plots}

Box plots are useful for visualizing the central tendency and spread of
continuous data, particularly when comparing distributions across
multiple groups.

\begin{figure}

\centering{

\includegraphics[width=0.7\textwidth,height=\textheight]{images/boxplot.png}

}

\caption{\label{fig-boxplot}Anatomy of a (horizontal) boxplot.}

\end{figure}%

This type of graph uses boxes and lines to represent the distributions.
In Figure~\ref{fig-boxplot} the box boundaries indicate the
interquartile range (IQR), covering the middle 50\% of the data, with a
horizontal line inside the box representing the median. Whiskers extend
from the box to capture the range of the remaining data, providing
additional insight into the spread. Data points lying outside the
whiskers are displayed as individual dots and are considered potential
outliers.

~

\begin{figure}

\begin{minipage}{0.50\linewidth}

\centering{

\includegraphics{descriptive_files/figure-pdf/fig-boxplot1-1.pdf}

}

\subcaption{\label{fig-boxplot1-1}Vertical box plot of age.}

\end{minipage}%
%
\begin{minipage}{0.50\linewidth}

\centering{

\includegraphics{descriptive_files/figure-pdf/fig-boxplot1-2.pdf}

}

\subcaption{\label{fig-boxplot1-2}Vertical box plot of time spent.}

\end{minipage}%

\caption{\label{fig-boxplot1}Box plots of age and time spent variables.}

\end{figure}%

\begin{tcolorbox}[enhanced jigsaw, bottomrule=.15mm, title={\includegraphics[width=1em,height=1em]{descriptive_files/figure-pdf/fa-icon-ee56651b74f8b687021aa43846e40a59.pdf}
Comment}, toprule=.15mm, opacitybacktitle=0.6, coltitle=black, opacityback=0, rightrule=.15mm, leftrule=.75mm, colback=white, bottomtitle=1mm, toptitle=1mm, colframe=quarto-callout-tip-color-frame, arc=.35mm, breakable, titlerule=0mm, left=2mm, colbacktitle=quarto-callout-tip-color!10!white]

We have already discussed outliers---observations that exhibit unusually
large or small values compared to the rest of the dataset. Outliers can
arise from several sources: they may result from incorrect measurements,
such as data entry errors or instrument malfunctions, or they may
originate from a different population than the rest of the data,
indicating a ``rare'' event. Understanding the causes of outliers is
essential for proper data analysis and interpretation.

\vspace{10pt}

We typically \textbf{identify outliers using the interquartile range}
(IQR), where any value outside the interval (Q1 - 1.5 * IQR, Q3 + 1.5 *
IQR) is considered a potential outlier (Tukey's method). This interval
extends 1.5 times the IQR below the first quartile (Q1) and above the
third quartile (Q3).

\end{tcolorbox}

\part{Jamovi LAB}

\chapter{LAB I: Introduction to Jamovi and data
preparation}\label{sec-lab1}

\begin{tcolorbox}[enhanced jigsaw, bottomrule=.15mm, title={\includegraphics[width=1em,height=1em]{lab1_files/figure-pdf/fa-icon-fd658532f1071e4d71f26381511f9b57.pdf}
Learning objectives}, toprule=.15mm, opacitybacktitle=0.6, coltitle=black, opacityback=0, rightrule=.15mm, leftrule=.75mm, colback=white, bottomtitle=1mm, toptitle=1mm, colframe=quarto-callout-caution-color-frame, arc=.35mm, breakable, titlerule=0mm, left=2mm, colbacktitle=quarto-callout-caution-color!10!white]

\begin{itemize}
\tightlist
\item
  Navigate Jamovi and import datasets.
\item
  Filter rows in the data for focused analysis.
\item
  Transform existing variables to prepare data for analysis.
\item
  Compute new variables based on other variables in the dataset.
\end{itemize}

\end{tcolorbox}

\section{Why Jamovi?}\label{why-jamovi}

Jamovi is a new fee open ``3rd generation'' statistical software that is
built on top of the programming language R (Figure~\ref{fig-jamovi_0}).
Designed from the ground up to be easy to use, Jamovi is a compelling
alternative to costly statistical products such as SPSS and SAS.

\begin{figure}

\centering{

\includegraphics[width=0.95\textwidth,height=\textheight]{images/jamovi_0.png}

}

\caption{\label{fig-jamovi_0}Jamovi is free and open statistical
software}

\end{figure}%

\begin{tcolorbox}[enhanced jigsaw, bottomrule=.15mm, title={\textbf{Some other advantages are:}}, toprule=.15mm, opacitybacktitle=0.6, coltitle=black, opacityback=0, rightrule=.15mm, leftrule=.75mm, colback=white, bottomtitle=1mm, toptitle=1mm, colframe=quarto-callout-note-color-frame, arc=.35mm, breakable, titlerule=0mm, left=2mm, colbacktitle=quarto-callout-note-color!10!white]

\begin{enumerate}
\def\labelenumi{\arabic{enumi}.}
\tightlist
\item
  User-friendly point-and-click interface.
\item
  Displays informative tables and clear visuals.
\item
  Supports add-on modules for advanced statistical analysis.
\item
  Allows integration with R.
\item
  Provides access to a user guide and community resources on the Jamovi
  website.
\end{enumerate}

\end{tcolorbox}

\section{Downloading and installing
Jamovi}\label{downloading-and-installing-jamovi}

\textbf{Jamovi} is available for Windows (64-bit), macOS, Linux and
ChromeOS. Installation on desktop is quite straight-forward. Just go to
the Jamovi download page \url{https://www.jamovi.org/download.html}, and
download the latest version (current release) for your operating system.

\section{Navigating Jamovi}\label{navigating-jamovi}

When jamovi first opens, we will see a screen something like in
Figure~\ref{fig-jamovi1}.

\begin{figure}

\includegraphics[width=0.95\textwidth,height=\textheight]{images/jamovi1.png}

\caption{\label{fig-jamovi1}Jamovi starts up!.}

\end{figure}%

To the left is the spreadsheet view, and to the right is where the
results of statistical tests appear. Down the middle is a bar separating
these two regions, and this can be dragged to the left or the right to
change their sizes.

~

Let's take a quick look at the Jamovi Main Menu, referred to hereafter
as the \textbf{Menu}, as shown in Figure~\ref{fig-menu1}. This Menu is
displayed at the very top of the Jamovi screen:

\begin{figure}

\includegraphics[width=0.95\textwidth,height=\textheight]{images/menu1.png}

\caption{\label{fig-menu1}The menu bar provides access to all functions
of the program.}

\end{figure}%

There are six tabs in the Menu (from left to right): 1. File (a layer
with three horizontal levels \(\equiv\)), 2. Variables, 3. Data, 4.
Analyses, 5. Edit, and 6. Settings (the three dots \(\vdots\) at the top
right of the window) tabs. A toolbar appears whenever we click on a Menu
tab (Table~\ref{tbl-menu}).

~

\begin{longtable}[]{@{}
  >{\raggedright\arraybackslash}p{(\columnwidth - 2\tabcolsep) * \real{0.5130}}
  >{\raggedright\arraybackslash}p{(\columnwidth - 2\tabcolsep) * \real{0.4870}}@{}}
\caption{Menu and toolbars of Jamovi}\label{tbl-menu}\tabularnewline
\toprule\noalign{}
\begin{minipage}[b]{\linewidth}\raggedright
Menu tab
\end{minipage} & \begin{minipage}[b]{\linewidth}\raggedright
Toolbar
\end{minipage} \\
\midrule\noalign{}
\endfirsthead
\toprule\noalign{}
\begin{minipage}[b]{\linewidth}\raggedright
Menu tab
\end{minipage} & \begin{minipage}[b]{\linewidth}\raggedright
Toolbar
\end{minipage} \\
\midrule\noalign{}
\endhead
\bottomrule\noalign{}
\endlastfoot
\begin{minipage}[t]{\linewidth}\raggedright
\begin{enumerate}
\def\labelenumi{\arabic{enumi}.}
\item
  \textbf{File tab} (\(\equiv\))

  The file tab
  \includegraphics[width=0.25in,height=0.17708in]{images/hbars.png}
  allows us to open/import existing files, save and export our files.
\end{enumerate}
\end{minipage} &
\includegraphics[width=0.4\textwidth,height=\textheight]{images/tab_file.png} \\
\begin{minipage}[t]{\linewidth}\raggedright
\begin{enumerate}
\def\labelenumi{\arabic{enumi}.}
\setcounter{enumi}{1}
\item
  \textbf{Variables tab}

  This allows us to view and search our variables in a list view.
\end{enumerate}
\end{minipage} & \begin{minipage}[t]{\linewidth}\raggedright
\includegraphics[width=1\textwidth,height=\textheight]{images/tab_variables.png}

This view allows us to easily navigate our variables and do the
following:

\begin{itemize}
\item
  Search for a variable by scrolling through the list or search for one
  by name.
\item
  Edit the variable names and descriptions by double-clicking in the
  relevant field.
\item
  Edit our variable details by double-clicking on the data symbol (the
  screen will appear for us to add all the necessary information).
\item
  Create a new variable by clicking on
  the\includegraphics[width=0.35417in,height=\textheight]{images/plus_icon.png}
  in the bottom right corner. \textbar{}
\end{itemize}
\end{minipage} \\
\begin{minipage}[t]{\linewidth}\raggedright
\begin{enumerate}
\def\labelenumi{\arabic{enumi}.}
\setcounter{enumi}{2}
\item
  \textbf{Data tab}

  Here we will see our raw data which are organised like Excel in rows
  and columns. We can also manipulate our data and add new variables
  when necessary.
\end{enumerate}
\end{minipage} & \begin{minipage}[t]{\linewidth}\raggedright
\includegraphics[width=1\textwidth,height=\textheight]{images/tab_data.png}

Specifically, this tab allows us to do the following:

\begin{itemize}
\item
  Rename and add details to existing variables. Click on the
  \emph{Setup} button, or double-click on the variable we want to
  manage.
\item
  Compute and transform variables
\item
  Add and/or Delete variables (columns)
\item
  Add Filters
\item
  Add and/or Delete Rows
\end{itemize}
\end{minipage} \\
\begin{minipage}[t]{\linewidth}\raggedright
\begin{enumerate}
\def\labelenumi{\arabic{enumi}.}
\setcounter{enumi}{3}
\item
  \textbf{Analyses tab}

  It includes the available statistical analyses that can be performed
  by Jamovi.
\end{enumerate}
\end{minipage} & \begin{minipage}[t]{\linewidth}\raggedright
\includegraphics{images/tab_analysis.png}

We will spend most of our time in the Analyses Tab. The following six
modules are pre-installed:

\begin{itemize}
\item
  Exploration
\item
  T-Tests
\item
  ANOVA
\item
  Regression
\item
  Frequencies
\item
  Factor
\end{itemize}

For example, if we want to perform regression analysis, we simply click
the '\,'Regression'' button.

All other modules need to be installed using the \textbf{Modules} button
(Plus button) in our top-right
\includegraphics[width=0.45833in,height=\textheight]{images/modules.png}
\textbar{}
\end{minipage} \\
\begin{minipage}[t]{\linewidth}\raggedright
\begin{enumerate}
\def\labelenumi{\arabic{enumi}.}
\setcounter{enumi}{4}
\item
  \textbf{Edit tab}

  It includes a toolbar similar to a word processor.
\end{enumerate}
\end{minipage} & \includegraphics{images/tab_edit.png}

We can add extra information to our results using the buttons that are
very similar to what we would find in Word (though there are fewer
options). \\
\begin{minipage}[t]{\linewidth}\raggedright
\begin{enumerate}
\def\labelenumi{\arabic{enumi}.}
\setcounter{enumi}{5}
\item
  \textbf{Settings tab}

  (the three dots \({\vdots}\) at the top right of the window)

  \includegraphics[width=0.14583in,height=0.30208in]{images/dots.png}

  It includes the application settings that can be manged by the users
  according to their preferences.
\end{enumerate}
\end{minipage} & \begin{minipage}[t]{\linewidth}\raggedright
\includegraphics[width=1.77083in,height=\textheight]{images/prefrences.png}

We can apply our preferences for a number of settings such as:

\begin{itemize}
\item
  How many decimal numbers we want.
\item
  If we want to learn R, we can also display the R syntax.
\item
  Our graph color scheme.
\item
  Our default missing value.
\end{itemize}
\end{minipage} \\
\end{longtable}

~

\section{Types of Variables in
Jamovi}\label{types-of-variables-in-jamovi}

Data variables can be one of four \textbf{measure types}:

\begin{itemize}
\item
  \includegraphics[width=0.25in,height=0.22917in]{images/nominal.png}
  \textbf{Nominal}: This type is for nominal categorical variables.
\item
  \includegraphics[width=0.27083in,height=\textheight]{images/ordinal.png}
  \textbf{Ordinal}: This type is for ordinal categorical variables.
\item
  \includegraphics[width=0.23958in,height=0.21875in]{images/continuous.png}
  \textbf{Continuous} : this type is for variables with numeric values
  which are considered to be of \emph{Interval} or \emph{Ratio} scales.
\item
  \includegraphics[width=0.25in,height=\textheight]{images/id.png}
  \textbf{ID}: This will usally be our first column. This can be text or
  numbers, but it should be unique to each row.
\end{itemize}

~

Additionally, data variables can be one of three \textbf{data types}:

\begin{itemize}
\item
  \textbf{Integer}: These are full numbers e.g.~1, 2, 3, \ldots{} 100,
  etc. - Integers can be used for all three measure types . When used
  for Nominal/Ordinal data numbers will represent labels e.g.~male=1;
  female=2.
\item
  \textbf{Decimal:} These are numbers with decimal points. e.g.~1.3,
  5.6, 7.8, etc. - This will usually only be used for continuous data.
\item
  \textbf{Text:} This can be used for ordinal and nominal data.
\end{itemize}

The measure types are designated by the symbol in the header of the
variable's column. Note that some combinations of data-type and
measure-type don't make sense, and Jamovi won't let us choose these.

\begin{longtable}[]{@{}
  >{\raggedright\arraybackslash}p{(\columnwidth - 6\tabcolsep) * \real{0.2222}}
  >{\raggedright\arraybackslash}p{(\columnwidth - 6\tabcolsep) * \real{0.2361}}
  >{\raggedright\arraybackslash}p{(\columnwidth - 6\tabcolsep) * \real{0.2361}}
  >{\raggedright\arraybackslash}p{(\columnwidth - 6\tabcolsep) * \real{0.2361}}@{}}
\caption{Types of data and measures}\label{tbl-types}\tabularnewline
\toprule\noalign{}
\begin{minipage}[b]{\linewidth}\raggedright
\end{minipage} & \begin{minipage}[b]{\linewidth}\raggedright
\end{minipage} & \begin{minipage}[b]{\linewidth}\raggedright
Measure
\end{minipage} & \begin{minipage}[b]{\linewidth}\raggedright
\end{minipage} \\
\midrule\noalign{}
\endfirsthead
\toprule\noalign{}
\begin{minipage}[b]{\linewidth}\raggedright
\end{minipage} & \begin{minipage}[b]{\linewidth}\raggedright
\end{minipage} & \begin{minipage}[b]{\linewidth}\raggedright
Measure
\end{minipage} & \begin{minipage}[b]{\linewidth}\raggedright
\end{minipage} \\
\midrule\noalign{}
\endhead
\bottomrule\noalign{}
\endlastfoot
\textbf{Data} & Nominal & Ordinal & Continuous \\
Integer & \({\checkmark}\) & \({\checkmark}\) & \({\checkmark}\) \\
Decimal & & & \({\checkmark}\) \\
Text & \({\checkmark}\) & \({\checkmark}\) & \\
\end{longtable}

~

\section{Importing data}\label{importing-data}

\subsection{The dataset}\label{the-dataset}

It is possible to simply begin typing values into the Jamovi spreadsheet
as we would with any other spreadsheet software. Alternatively, existing
datasets in a range of formats (OMV, Excel, CSV, SPSS, R data, Stata,
SAS) can be opened in Jamovi. We will use the following dataset as an
example (Figure~\ref{fig-rses}).

\begin{verbatim}
file:///C:/Users/kboug/AppData/Local/Temp/Rtmp88kkK1/file42fc662e71f4/widget42fc48d4b4d.html screenshot completed
\end{verbatim}

\begin{figure}

\centering{

\includegraphics{lab1_files/figure-pdf/fig-rses-1.pdf}

}

\caption{\label{fig-rses}Table with raw data.}

\end{figure}%

(\textbf{NOTE:} You can find other formats of the data (OMV, Excel) at
the link: \url{https://osf.io/gvctz/}).

~ ~

The meta-data (data about the data) for this dataset are as following:

\begin{enumerate}
\def\labelenumi{\arabic{enumi}.}
\tightlist
\item
  \textbf{sex:} sex (1 = male, 2 = female).
\item
  \textbf{age:} age in years.
\item
  \textbf{time\_spend:} hours spent on social media.
\item
  \(q_1 ... q_{10}\): Ten questions (items) of Rosenberg Self-Esteem
  Scale (RSES). The 10 items are answered on a four point scale ranging
  from strongly agree to strongly disagree coded as follows: Strongly
  Agree = 3, Agree = 2, Disagree = 1, and Strongly Disagree = 0.
\end{enumerate}

\begin{itemize}
\item
  Positively worded Items 1, 2, 4, 6, and 7.
\item
  Negatively worded Items 3, 5, 8, 9, and 10 (codes should be reversed).
\end{itemize}

\begin{figure}

\centering{

\includegraphics{images/rsescode.png}

}

\caption{\label{fig-rsescode}Codes should be reversed for the negatively
worded Items.}

\end{figure}%

The scale ranges from 0-30 (we add the scores for all items), with 30
indicating the highest total score possible.

More information for the RSES:
\href{https://socy.umd.edu/about-us/using-rosenberg-self-esteem-scale}{Rosenberg
Self-Esteem Scale}.

\subsection{Opening the file}\label{opening-the-file}

To open this \texttt{csv} file, click on the \textbf{File} tab
\includegraphics[width=0.29167in,height=0.17708in]{images/hbars.png} at
the top left hand corner (just left of the \textbf{Variables} tab)
(Figure~\ref{fig-tab_analysis_file}).

\begin{figure}

\centering{

\includegraphics{images/tab_analysis_file.png}

}

\caption{\label{fig-tab_analysis_file}Click on the File tab}

\end{figure}%

This will open the menu shown in Figure~\ref{fig-open_file}. Select
`\textbf{Open}' and then `\textbf{This PC}'. Choose the downloaded file
from the files listed on `\textbf{Browse}' which are stored on our
computer folders:

\begin{figure}

\includegraphics[width=0.95\textwidth,height=\textheight]{images/open_file.png}

\caption{\label{fig-open_file}Open an existing file stored on our
computer into Jamovi.}

\end{figure}%

~

The flowchart of the process is:

\includegraphics[width=6.62in,height=0.72in]{lab1_files/figure-latex/mermaid-figure-1.png}

~

We should see data now in the Spreadsheet view
(Figure~\ref{fig-dataset}).

\begin{figure}

\includegraphics[width=0.95\textwidth,height=\textheight]{images/dataset.png}

\caption{\label{fig-dataset}Our dataset.}

\end{figure}%

As we can see this is a data set with 258 observations and 13 variables.
JAMOVI has classified all the variables as nominal
\includegraphics[width=0.25in,height=0.22917in]{images/nominal.png};
however, only the \texttt{sex} variable is actually nominal.

\section{Adding labels to codes}\label{adding-labels-to-codes}

\begin{itemize}
\tightlist
\item
  \textbf{sex variable}
\end{itemize}

The first variable, named \texttt{sex}, is a categorical variable coded
as 1 for males and 2 for females. Notice that it is correctly identified
as a nominal
\includegraphics[width=0.25in,height=0.22917in]{images/nominal.png}
variable in Jamovi.

We can assign labels to numerically coded values of categorical
variables, such as \texttt{sex}, by accessing the data variable
settings. One way to achieve this is by double-clicking on the variable
name \texttt{sex}, which opens the additional menu of variable settings
at the top of the Jamovi screen (Figure~\ref{fig-labels0}).

\begin{figure}

\centering{

\includegraphics{images/labels0.png}

}

\caption{\label{fig-labels0}An additional menu appears at the top of the
Jamovi screen after double-clicking on the variable name sex.}

\end{figure}%

~

In this menu, we will find the \emph{Levels} setup. Here, we can specify
the labels that should appear for each category level. Click on the
number ``1'' in the \emph{Levels} box to edit its label, changing it
from ``1'' to ``male''. Similarly, click on the number ``2'' and change
it to ``female'' (Figure~\ref{fig-labels}).

\begin{figure}

\centering{

\includegraphics{images/labels.png}

}

\caption{\label{fig-labels}Adding labels to numerically coded Values.}

\end{figure}%

Notice how the numbers ``1'' and ``2'' have moved to the lower right
under the text we're typing, allowing us to still see which label
corresponds to each numerical code. Press \texttt{Enter} or click
anywhere outside the labels box to save these labels.

We close the variable settings by pressing the arrow in the top-right
corner
\includegraphics[width=0.26042in,height=0.26042in]{images/up_arrow.png}.

\section{Changing the measure type}\label{changing-the-measure-type}

Next, we will change the measure type of \texttt{age} and
\texttt{time\_spent} variables from nominal to continuous.

\begin{itemize}
\tightlist
\item
  \textbf{age variable}
\end{itemize}

Double-click on the variable name \texttt{age} to open the data variable
settings, as shown in Figure~\ref{fig-type_age}:

\begin{figure}

\centering{

\includegraphics[width=0.9\textwidth,height=\textheight]{images/type_age.png}

}

\caption{\label{fig-type_age}Data variable menu settings for the age
variable.}

\end{figure}%

~

From the drop-down list of ``Measure type'' we select the continuous
type
\includegraphics[width=0.23958in,height=0.21875in]{images/continuous.png},
as shown in Figure~\ref{fig-type_age2}.

\begin{figure}

\centering{

\includegraphics[width=0.8\textwidth,height=\textheight]{images/type_age2.png}

}

\caption{\label{fig-type_age2}Change the measure type of age from
nominal to continuous. .}

\end{figure}%

~

\begin{itemize}
\tightlist
\item
  \textbf{time\_spent variable}
\end{itemize}

Instead of closing the data variable menu using the arrow in the
top-right corner, we can click on the
\includegraphics[width=0.26042in,height=0.26042in]{images/greater.png}
to proceed to the next variable setting, \texttt{time\_spent}. As
before, we select the continuous type for this variable from the
``Measure type'' drop-down list, as shown in Figure~\ref{fig-time}.

\begin{figure}

\centering{

\includegraphics[width=0.8\textwidth,height=\textheight]{images/time.png}

}

\caption{\label{fig-time}Change the measure type of time\_spent from
nominal to continuous.}

\end{figure}%

We close the variable settings by pressing the arrow in the top-right
corner
\includegraphics[width=0.26042in,height=0.26042in]{images/up_arrow.png}.

~

\begin{itemize}
\tightlist
\item
  \textbf{q1 to q10 variables}
\end{itemize}

Finally, we will change the measure type of \texttt{q1} to \texttt{q10}
variables from nominal to ordinal. In the Variables tab, we select the
checkboxes for the \texttt{q1} through \texttt{q10} variables, as shown
in Figure~\ref{fig-q1_10}.

\begin{figure}

\centering{

\includegraphics[width=0.6\textwidth,height=\textheight]{images/q1_10.png}

}

\caption{\label{fig-q1_10}Change the measure type of time\_spent from
nominal to continuous.}

\end{figure}%

~

After that, we click on the Edit button
\includegraphics[width=0.375in,height=0.41667in]{images/edit.png} to
open the data variable settings, as shown in
Figure~\ref{fig-multiple_q1_10}.

\begin{figure}

\centering{

\includegraphics{images/multiple_q1_10.png}

}

\caption{\label{fig-multiple_q1_10}Change the measure type of
time\_spent from nominal to continuous.}

\end{figure}%

~

From the drop-down list of ``Measure type'' we select the ordinal type
\includegraphics[width=0.23958in,height=0.21875in]{images/ordinal.png},
as shown in Figure~\ref{fig-ordinal_multipleq1_10}.

\begin{figure}

\centering{

\includegraphics[width=0.8\textwidth,height=\textheight]{images/ordinal_multipleq1_10.png}

}

\caption{\label{fig-ordinal_multipleq1_10}Change the measure type of q1
to q10 from nominal to ordinal.}

\end{figure}%

~

\section{Filtering rows}\label{filtering-rows}

Next, we select the \textbf{Filters button}
\includegraphics[width=0.3125in,height=0.29167in]{images/filters.png}
from the \textbf{Data} tab. This opens the ``Row FILTERS'' view at the
top of the Jamovi screen where we can add a filter called
``\textbf{Filter 1}'' (Figure~\ref{fig-filter1_age}). Let's say that we
want to study only the adults from the participants in this survey.

\begin{itemize}
\tightlist
\item
  \textbf{Simple condition}
\end{itemize}

In order to access functions, press the
\includegraphics[width=0.3125in,height=0.29167in]{images/fx.png} icon in
the filter settings and from ``\textbf{VARIABLES}'' double-click on
\texttt{age} (or we just type the variable name but if the variable has
a space, we must use ticks \texttt{\textquotesingle{}\textquotesingle{}}
around the variable name). Then type the condition
\texttt{age\ \textgreater{}=\ 18} in the formula box and press
\textbf{ENTER} from the keyboard (Figure~\ref{fig-filter1_age}).

\begin{figure}

\centering{

\includegraphics{images/filter1_age.png}

}

\caption{\label{fig-filter1_age}Adding a filter.}

\end{figure}%

~

\begin{tcolorbox}[enhanced jigsaw, bottomrule=.15mm, title=\textcolor{quarto-callout-tip-color}{\faLightbulb}\hspace{0.5em}{Relational (or comparison) operators in Jamovi}, toprule=.15mm, opacitybacktitle=0.6, coltitle=black, opacityback=0, rightrule=.15mm, leftrule=.75mm, colback=white, bottomtitle=1mm, toptitle=1mm, colframe=quarto-callout-tip-color-frame, arc=.35mm, breakable, titlerule=0mm, left=2mm, colbacktitle=quarto-callout-tip-color!10!white]

\begin{longtable}[]{@{}ll@{}}
\toprule\noalign{}
\textbf{symbol} & \textbf{read as} \\
\midrule\noalign{}
\endhead
\bottomrule\noalign{}
\endlastfoot
\textless{} & less than \\
\textgreater{} & greater than \\
== & equal to \\
\textless= & less than or equal to \\
\textgreater= & greater than or equal to \\
!= & not equal to \\
\end{longtable}

\end{tcolorbox}

~

Notice that a column named \texttt{Filter\ 1} has been added to the
Spreadsheet view. Cells that meet the condition
\texttt{age\ \textgreater{}=\ 18} are checked with a green tick
\includegraphics[width=0.25in,height=0.21875in]{images/tick.png}, while
the rest have a red x symbol
\includegraphics[width=0.25in,height=0.21875in]{images/x.png}. Lines
with an \texttt{X} are grayed out indicating that these observations are
now outside of the current dataset.

There is also a switch where we can activate
\includegraphics[width=0.28125in,height=0.20833in]{images/active.png} or
inactivate
\includegraphics[width=0.28125in,height=0.20833in]{images/inactive.png}
the filter (note that an inactivate filter will remain visible and can
be toggled to active at any time).

It is also possible to hide all filter columns by clicking on the eye
symbol
\includegraphics[width=0.27083in,height=0.23958in]{images/eye.png} of
the filters. In this case, all filters and the filtered data will remain
active but will be invisible.

Finally, if we want to delete the filter permanently, we can click on
the \includegraphics[width=0.28125in,height=0.20833in]{images/x_del.png}
of the filter .

~

\begin{itemize}
\tightlist
\item
  \textbf{Multiple conditions}
\end{itemize}

But we can do more complicated filters than this! Let's say that we're
interested in the \textbf{adult females}. In fact we can specify this in
three ways:

\textbf{a)} by using the \texttt{and} operator in ``\textbf{Filter 1}''
which means that both conditions (adults and females) in the expression
must be true at the same time. Therefore, we type the expression
\texttt{age\ \textgreater{}=\ 18\ and\ sex\ ==\ \textquotesingle{}female\textquotesingle{}}
(Figure~\ref{fig-filter1_age_and_sex}):

\begin{figure}

\centering{

\includegraphics{images/filter1_age_and_sex.png}

}

\caption{\label{fig-filter1_age_and_sex}Combining conditions with
``and'' operator in one expression.}

\end{figure}%

Note that in the above expression we can also use double quotes around
the \texttt{"female"} (i.e.,
\texttt{age\ \textgreater{}=\ 18\ and\ sex\ ==\ "female"}) or even the
coded value (i.e., \texttt{age\ \textgreater{}=\ 18\ and\ sex\ ==\ 2}).

\textbf{b)} by adding the second condition
(i.e.~\texttt{sex\ ==\ "female"}) as \textbf{another expression} to
``\textbf{Filter 1}'' (by clicking the small \texttt{+} beside the first
expression) (Figure~\ref{fig-filter1_age_sex}):

\begin{figure}

\centering{

\includegraphics{images/filter1_age_sex.png}

}

\caption{\label{fig-filter1_age_sex}Multiple expressions in the same
filter.}

\end{figure}%

This additional expression comes to be represented with its own column
\texttt{F1(2)} (Figure~\ref{fig-filter1_age_sex}), and by looking at the
ticks and crosses, we can see which expression is responsible for
excluding each row.

\textbf{c)} adding a new ``\textbf{Filter 2}'' (by selecting the
\textbf{large +} to the left of the filters dialog box)
(Figure~\ref{fig-filter2}). In this case, we can activate or inactivate
the filters separately.

\begin{figure}

\centering{

\includegraphics{images/filter2.png}

}

\caption{\label{fig-filter2}Multiple expressions using multiple
filters.}

\end{figure}%

We close the filter settings by pressing the arrow in the top-right
corner
\includegraphics[width=0.26042in,height=0.26042in]{images/up_arrow.png}.

\begin{tcolorbox}[enhanced jigsaw, bottomrule=.15mm, title=\textcolor{quarto-callout-important-color}{\faExclamation}\hspace{0.5em}{Important}, toprule=.15mm, opacitybacktitle=0.6, coltitle=black, opacityback=0, rightrule=.15mm, leftrule=.75mm, colback=white, bottomtitle=1mm, toptitle=1mm, colframe=quarto-callout-important-color-frame, arc=.35mm, breakable, titlerule=0mm, left=2mm, colbacktitle=quarto-callout-important-color!10!white]

Filters in Jamovi exclude the rows for which the expression is not true.
When filters are active, \textbf{all results} will be based on the
filtered data. If we want to see unfiltered results we will either need
to delete the filter or toggle it to inactive.

\end{tcolorbox}

~

\section{Transforming a variable}\label{transforming-a-variable}

\subsection{Transform a quantitative variable into a qualitative
variable}\label{transform-a-quantitative-variable-into-a-qualitative-variable}

Sometimes it can be useful to convert a quantitative variable into a
qualitative variable with levels. For example, the \texttt{time\_spent}
can be categorized as follows:

\begin{itemize}
\item
  less than or equal to 3
  \includegraphics[width=0.88em,height=1em]{lab1_files/figure-pdf/fa-icon-21c27fa153c8e039995b178d5d474e3f.pdf}
  ``0-3''
\item
  4 to 7
  \includegraphics[width=0.88em,height=1em]{lab1_files/figure-pdf/fa-icon-21c27fa153c8e039995b178d5d474e3f.pdf}
  ``4-7''
\item
  8 to 11
  \includegraphics[width=0.88em,height=1em]{lab1_files/figure-pdf/fa-icon-21c27fa153c8e039995b178d5d474e3f.pdf}
  ``8-11''
\item
  greater than 11
  \includegraphics[width=0.88em,height=1em]{lab1_files/figure-pdf/fa-icon-21c27fa153c8e039995b178d5d474e3f.pdf}
  ``\textgreater11''
\end{itemize}

Select \texttt{time\_spent} variable and click the Transform button
\includegraphics[width=0.41667in,height=0.375in]{images/transform_button.png}
from the toolbar of the Data tab. This opens the ``\textbf{TRANSFORMED
VARIABLE}'' view, where we can set the name of the transformed variable
such as \texttt{time\_spent2} and create the transformation. To do this,
select the source variable, the \texttt{time\_spent} in this case, in
the \emph{Source Variable} field and create a new transformation using
transform field (Figure~\ref{fig-create_new_transf}), which is initially
set to `none'.

\begin{figure}

\centering{

\includegraphics{images/create_new_transf.png}

}

\caption{\label{fig-create_new_transf}Setting up a new transformation of
a variable.}

\end{figure}%

~

This opens the ``TRANSFORM'' view where Jamovi gives each transformation
a name (e.g., \texttt{Transform\ 1}; if we want we can change it). This
allows us to use it again later on other variables if we wish. We can
also add a description (Figure~\ref{fig-transf_view}).

\begin{figure}

\centering{

\includegraphics[width=0.6\textwidth,height=\textheight]{images/transf_view.png}

}

\caption{\label{fig-transf_view}Box for adding conditions to the
transformation.}

\end{figure}%

Now we need to \textbf{add conditions}. Jamovi uses simple
\texttt{if\ ...\ else} statements and executes each statement starting
from the top. So let's start!

First, select \textbf{\texttt{+\ Add\ recode\ condition}}. Second, we
need to fill the \textbf{boxes} with the information as follows
(Figure~\ref{fig-recode1}):

\begin{itemize}
\tightlist
\item
  The \textbf{\texttt{\$source}} is the variable we want to transform
  (here \texttt{time\_spent})- don't change this.
\item
  Select the appropriate \textbf{comparison operator} (here
  \texttt{\textless{}=} )
\item
  In the next box, we will put the \texttt{time\_spent} value we want as
  the cut off point (e.g., 3).
\item
  After the use, add our new label (here
  \texttt{\textquotesingle{}0-3\textquotesingle{}}). If we are using
  text we must enclose it in
  \texttt{\textquotesingle{}...\textquotesingle{}}
\end{itemize}

\begin{figure}

\centering{

\includegraphics{images/recode1.png}

}

\caption{\label{fig-recode1}Adding the first condition (0-3).}

\end{figure}%

~

We can add as many conditions as we want by selecting
\texttt{+\ Add\ recode\ condition}. This will add a new
\texttt{if\ \$source} line into the box (Figure~\ref{fig-recode2}).
Remember they will be executed in order.

\begin{figure}

\centering{

\includegraphics{images/recode2.png}

}

\caption{\label{fig-recode2}Adding the second condition (4-7).}

\end{figure}%

~

\begin{figure}

\centering{

\includegraphics{images/recode3.png}

}

\caption{\label{fig-recode3}Adding the third condition (8-11).}

\end{figure}%

~

Finally, after the \texttt{else\ use} box just add the label for the
data that does not meet the above conditions (here
\texttt{\textquotesingle{}\textgreater{}11\textquotesingle{}}), as shown
in Figure~\ref{fig-recode4}.

\begin{figure}

\centering{

\includegraphics{images/recode4.png}

}

\caption{\label{fig-recode4}Adding the final label for the data that
does not meet the above conditions (\textgreater11).}

\end{figure}%

~

\subsection{Reverse scoring of items}\label{reverse-scoring-of-items}

To compute the total score for all items of the RSES, we first need to
reverse the scores of the negatively worded questions (3, 5, 8, 9, and
10). To do this, we can follow these simple steps.

In the Variables tab, we select the checkboxes for the \texttt{q3},
\texttt{q5}, \texttt{q8}, \texttt{q9}, and \texttt{q10} variables, as
shown in Figure~\ref{fig-rev_items}.

\begin{figure}

\centering{

\includegraphics[width=0.6\textwidth,height=\textheight]{images/rev_items.png}

}

\caption{\label{fig-rev_items}Select the variables to be reversed.}

\end{figure}%

~

After that, we click on the Transform button
\includegraphics[width=0.41667in,height=0.375in]{images/transform_button.png})
from the toolbar of the Variables tab to open the Transform variable
settings. Then, we \texttt{Create\ New\ Transform}, as shown in
Figure~\ref{fig-new_transf}.

\begin{figure}

\centering{

\includegraphics[width=0.6\textwidth,height=\textheight]{images/new_transf.png}

}

\caption{\label{fig-new_transf}Open the Transform variable settings.}

\end{figure}%

~

In the Transform view, we enter \texttt{\_R} as a suffix for the
variable names and specify \texttt{3-\$source} in the condition box, as
shown in Figure~\ref{fig-transform2}.

\begin{figure}

\centering{

\includegraphics[width=0.6\textwidth,height=\textheight]{images/transform2.png}

}

\caption{\label{fig-transform2}Transform variable settings.}

\end{figure}%

\begin{figure}

\centering{

\includegraphics[width=0.6\textwidth,height=\textheight]{images/reversed_data.png}

}

\caption{\label{fig-reversed_data}Data with the reversed variables.}

\end{figure}%

~

\section{Compute a new variable}\label{compute-a-new-variable}

Finally, we compute each participant's total score (ranging from 0 to
30) based on their responses to the 10 questions of the Rosenberg
Self-Esteem Scale.

Adding computed variables to a Jamovi spreadsheet is straightforward.
Click on the Compute button
\includegraphics[width=0.41667in,height=0.375in]{images/compute_button.png})
from the toolbar of the Data tab. An empty column has been created at
the end of our dataset (Figure~\ref{fig-compute1}). The black dot symbol
in the right of the column header indicates that this is a computed
variable.

\begin{figure}

\centering{

\includegraphics[width=0.6\textwidth,height=\textheight]{images/compute1.png}

}

\caption{\label{fig-compute1}Compute a new variable.}

\end{figure}%

To set up the computed variable, either double-click the column header,
or click the \textbf{Setup} button
\includegraphics[width=0.35417in,height=0.375in]{images/setup_button.png})
in the \textbf{Data} tab. This opens the ``\textbf{COMPUTED VARIABLE}''
view, where we can name the computed variable \texttt{score}. Next, we
select the \texttt{SUM} function from the available options for use in
the \textbf{Formula} box. Finally, we add the \texttt{q}-variables in
the formula separated by comma (note that we must use the reversed
variables in the sum) and we press \texttt{ENTER}
(Figure~\ref{fig-sum_score}).

\begin{figure}

\centering{

\includegraphics{images/sum_score.png}

}

\caption{\label{fig-sum_score}Computation of the total score.}

\end{figure}%

\chapter{LAB IV: Sampling distribution and Confidence
Interval}\label{sec-lab4}

When we have finished this Lab, we should be able to:

\begin{tcolorbox}[enhanced jigsaw, bottomrule=.15mm, title={Learning objectives}, toprule=.15mm, opacitybacktitle=0.6, coltitle=black, opacityback=0, rightrule=.15mm, leftrule=.75mm, colback=white, bottomtitle=1mm, toptitle=1mm, colframe=quarto-callout-caution-color-frame, arc=.35mm, breakable, titlerule=0mm, left=2mm, colbacktitle=quarto-callout-caution-color!10!white]

\begin{itemize}
\tightlist
\item
  Know the basic properties of sampling distribution of mean
\item
  Understand the Central Limit Theorem (CLM)
\item
  Understand the concept of the confidence interval of mean
\end{itemize}

\end{tcolorbox}

~

\section{The Sampling Distribution of mean and the
CLT}\label{the-sampling-distribution-of-mean-and-the-clt}

In this Lab we will learn the Central Limit Theorem (CLT), which is the
basis for many statistical concepts. We are going to explore this
concept with the help of a \emph{Shiny} application. So, clink on the
following link \href{https://gallery.shinyapps.io/CLT_mean/}{CLM}.

A \emph{Shiny app} opens in a web window as shown below
(Figure~\ref{fig-shiny_app1}):

\begin{figure}

\centering{

\includegraphics{images/shiny_app1.png}

}

\caption{\label{fig-shiny_app1}The Shiny application simulating the
Central limit Theorem for Means}

\end{figure}%

To the left is the interactive panel with radio buttons and slider bars,
and to the right there are three tabs:

\begin{itemize}
\item
  Population Distribution.
\item
  Samples.
\item
  Sampling Distribution.
\end{itemize}

First we are asked to choose from a Normal, Uniform, Right Skewed or
Left Skewed \emph{Parent distribution (Population)} from the left panel.
Let's select \emph{Right skewed} and then \emph{High skew} from the drop
down menu with the name \emph{Skew}, as shown in
Figure~\ref{fig-shiny_app2}.

\begin{figure}

\centering{

\includegraphics{images/shiny_app2.png}

}

\caption{\label{fig-shiny_app2}The case of a high right skewed
population distribution}

\end{figure}%

Next we set the \emph{Sample size} slider bar to 5 and the \emph{Number
of samples} to 1000, then select the \emph{Samples} tab, as shown in
Figure~\ref{fig-shiny_app2_1}. The first eight samples randomly drawn
from the original distribution are demonstrated in the panel. For
example, in the first box labeled \emph{Sample 1}, we observe five data
points (the sample size we set), along with their sample mean and
standard deviation (highlighted in the red circle).

\begin{figure}

\centering{

\includegraphics{images/shiny_app2_1.png}

}

\caption{\label{fig-shiny_app2_1}First eight samples randomly drawn from
the original distribution.}

\end{figure}%

Finally, we select the \emph{Sampling distribution} tab, which displays
the distribution of the 1000 \textbf{sample means}. We observe that this
distribution is right skewed with mean approximately equal to population
mean (Figure~\ref{fig-shiny_app3}).

\begin{figure}

\centering{

\includegraphics{images/shiny_app3.png}

}

\caption{\label{fig-shiny_app3}Distribution of means of 1000 random
samples, each consisting of 5 observations from a high right skewed
population distribution}

\end{figure}%

Now, try to increase the \emph{sample size} to 30
(Figure~\ref{fig-shiny_app4}):

\begin{figure}

\centering{

\includegraphics{images/shiny_app4.png}

}

\caption{\label{fig-shiny_app4}Distribution of means of 1000 random
samples, each consisting of 30 observations from a high right skewed
population distribution}

\end{figure}%

and then increase it to 200 (Figure~\ref{fig-shiny_app5}):

\begin{figure}

\centering{

\includegraphics{images/shiny_app5.png}

}

\caption{\label{fig-shiny_app5}Distribution of means of 1000 random
samples, each consisting of 200 observations from a high right skewed
population distribution}

\end{figure}%

We observe that the sampling distribution becomes closer and closer to
Normal and the standard error of the mean, \emph{SE}, (the standard
deviation of sample means) gets smaller as the sample size increases.
The important point is that whatever the parent distribution of a
variable, the distribution of the sample means will be nearly Normal, as
long as the samples are large enough.

~

\begin{tcolorbox}[enhanced jigsaw, bottomrule=.15mm, title={Properties of the sampling distribution of the mean}, toprule=.15mm, opacitybacktitle=0.6, coltitle=black, opacityback=0, rightrule=.15mm, leftrule=.75mm, colback=white, bottomtitle=1mm, toptitle=1mm, colframe=quarto-callout-tip-color-frame, arc=.35mm, breakable, titlerule=0mm, left=2mm, colbacktitle=quarto-callout-tip-color!10!white]

\begin{itemize}
\tightlist
\item
  The mean of the sampling distribution is the same as the mean of the
  population.
\item
  The standard deviation of the sampling distribution (i.e., the
  standard error) gets smaller as the sample size increases.
\item
  \textbf{According to the Central Limit Theorem (CLM)}, the shape of
  the sampling distribution becomes normal as the sample size increases
  regardless of the variable's population distribution.
\end{itemize}

\end{tcolorbox}

\section{The confidence interval of
mean}\label{the-confidence-interval-of-mean}

We are going to explore the concept of confidence interval (CI) of mean
with the help of a \emph{Shiny} application. So, clink on the following
link \href{https://shiny.rit.albany.edu/stat/confidence/}{CIs}.

A \emph{Shiny} app opens in a web window as shown below
(Figure~\ref{fig-shiny_app6}):

\begin{figure}

\centering{

\includegraphics{images/shiny_app6.png}

}

\caption{\label{fig-shiny_app6}Shiny application that simulates the
concept of confidence interval (CI) of mean}

\end{figure}%

On the left is the interactive panel with radio buttons and drop down
menus, and to the right there are two tabs:

\begin{itemize}
\item
  Plots.
\item
  About.
\end{itemize}

We retain active the \emph{Plots} tab.

First we are asked to choose if we want the \emph{Confidence Interval
Graph only} or the \emph{Confidence Interval Graph Plus Sampling
Distribution of the Mean}. Let's select the first choice and set the
\emph{Number of Simulated Samples} to one and the \emph{Sample Size} to
10 from the drop down menus, as shown in Figure~\ref{fig-shiny_app7}. A
horizontal bar will be created which represents the confidence interval
(CI), centered on the sample mean (point). In this case, the 95\% CI for
the sample mean includes the true value of the population mean (it
crosses the solid vertical line) and it is drawn as a black line.

\begin{figure}

\centering{

\includegraphics{images/shiny_app7.png}

}

\caption{\label{fig-shiny_app7}Confidence Interval Graph with one random
sample of 10 observations selected from a normal population
distribution}

\end{figure}%

Now, try to increase the \emph{Number of Simulated Samples} to 100
(Figure~\ref{fig-shiny_app8}):

\begin{figure}

\centering{

\includegraphics{images/shiny_app8.png}

}

\caption{\label{fig-shiny_app8}Confidence Interval Graph with 100 random
samples, each consisting of 10 observations from a normal population
distribution}

\end{figure}%

We observe that 5 out of 100 confidence intervals (red horizontal lines)
do not include the true population mean (the solid vertical line)
(Figure~\ref{fig-shiny_app8}). This is what we would expect -- that the
95\% confidence interval will not include the true population mean 5\%
of the time.

~

Next, we create the confidence intervals of 100 randomly generated
samples of size = 50 from the population (Figure~\ref{fig-shiny_app9}):

\begin{figure}

\centering{

\includegraphics{images/shiny_app9.png}

}

\caption{\label{fig-shiny_app9}Confidence Interval Graph with 100 random
samples, each consisting of 50 observations from a normal population
distribution}

\end{figure}%

We observe that the sample means are closer to the true population mean
and the 95\% CIs of the mean become narrower
(Figure~\ref{fig-shiny_app9}) increasing the sample size.

\chapter{LAB VI: Inference for numerical data (2
samples)}\label{sec-lab6}

When we have finished this Lab, we should be able to:

\begin{tcolorbox}[enhanced jigsaw, bottomrule=.15mm, title={Learning objectives}, toprule=.15mm, opacitybacktitle=0.6, coltitle=black, opacityback=0, rightrule=.15mm, leftrule=.75mm, colback=white, bottomtitle=1mm, toptitle=1mm, colframe=quarto-callout-caution-color-frame, arc=.35mm, breakable, titlerule=0mm, left=2mm, colbacktitle=quarto-callout-caution-color!10!white]

\begin{itemize}
\tightlist
\item
  Applying hypothesis testing
\item
  Compare two independent samples
\item
  Compare paired (related) samples
\item
  Interpret the results
\end{itemize}

\end{tcolorbox}

\section{Two-sample t-test (Student's
t-test)}\label{two-sample-t-test-students-t-test}

Two sample t-test (Student's t-test) can be used if we have two
independent (unrelated) groups (e.g., males-females, treatment-non
treatment) and one quantitative variable of interest.

\subsection{Opening the file}\label{opening-the-file-1}

Open the dataset named \texttt{depression} from the file tab in the
menu:

\begin{figure}

\centering{

\includegraphics[width=0.85\textwidth,height=\textheight]{images/depression1.png}

}

\caption{\label{fig-depression1}The depression dataset}

\end{figure}%

The dataset \texttt{depression} includes 76 patients and has two
variables. The \texttt{treatment} variable and the \texttt{HDRS}
variable (Figure~\ref{fig-depression1}). Double-click on the variable
name \texttt{HDRS} and change the measure type from nominal
\includegraphics[width=0.23958in,height=0.21875in]{images/nominal.png}
to continuous
\includegraphics[width=0.23958in,height=0.21875in]{images/continuous.png}.

\subsection{Research question}\label{research-question}

In an experiment designed to test the effectiveness of paroxetine for
treating bipolar depression, the participants were randomly assigned
into two groups (intervention Vs placebo).

The researchers used the Hamilton Depression Rating Scale (HDRS) to
measure the depression state of the participants and wanted to find out
if the HDRS score is different in paroxetine group as compared to
placebo group at the end of the experiment. The significance level α was
set to 0.05.

Note A score of 0--7 in HDRS is generally accepted to be within the
normal range, while a score of 20 or higher indicates at least moderate
severity.

\subsection{Hypothesis Testsing for the Student's
t-test}\label{hypothesis-testsing-for-the-students-t-test}

\begin{tcolorbox}[enhanced jigsaw, bottomrule=.15mm, title={Null hypothesis and alternative hypothesis}, toprule=.15mm, opacitybacktitle=0.6, coltitle=black, opacityback=0, rightrule=.15mm, leftrule=.75mm, colback=white, bottomtitle=1mm, toptitle=1mm, colframe=quarto-callout-note-color-frame, arc=.35mm, breakable, titlerule=0mm, left=2mm, colbacktitle=quarto-callout-note-color!10!white]

\begin{itemize}
\tightlist
\item
  \(H_0\): the means of HDRS in the two groups are equal
  (\(\mu_{1} = \mu_{2}\))
\item
  \(H_1\): the means of HDRS in the two groups are not equal
  (\(\mu_{1} \neq \mu_{2}\))
\end{itemize}

\end{tcolorbox}

\subsection{Assumptions}\label{assumptions}

\begin{tcolorbox}[enhanced jigsaw, bottomrule=.15mm, title={Check if the following assumptions are satisfied}, toprule=.15mm, opacitybacktitle=0.6, coltitle=black, opacityback=0, rightrule=.15mm, leftrule=.75mm, colback=white, bottomtitle=1mm, toptitle=1mm, colframe=quarto-callout-note-color-frame, arc=.35mm, breakable, titlerule=0mm, left=2mm, colbacktitle=quarto-callout-note-color!10!white]

\begin{enumerate}
\def\labelenumi{\arabic{enumi}.}
\tightlist
\item
  The data are \textbf{normally} distributed in both groups
\item
  The data in both groups have similar \textbf{variance} (also named as
  homogeneity of variance or homoscedasticity)
\end{enumerate}

\end{tcolorbox}

\textbf{A. Explore the descriptive characteristics of distribution for
each group and check for normality}

The distributions can be explored visually with appropriate plots.
Additionally, summary statistics and significance tests to check for
normality (e.g., Shapiro-Wilk test) and for equality of variances (e.g.,
Levene's test) can be used.

On the Jamovi top menu navigate to

\includegraphics[width=4.06in,height=0.52in]{lab6_files/figure-latex/mermaid-figure-1.png}

as shown below in Figure~\ref{fig-hdrs0}.

\begin{figure}

\centering{

\includegraphics[width=0.75\textwidth,height=\textheight]{images/hdrs0.png}

}

\caption{\label{fig-hdrs0}In the Analyses Tab select Exploration and
click on Descriptives.}

\end{figure}%

The \texttt{Descriptives} dialogue box opens. Drag the variable
\texttt{HDRS} into the \texttt{Variables} box and split it by the
\texttt{treatment} variable, as shown below
(Figure~\ref{fig-descriptives12}):

\begin{figure}

\centering{

\includegraphics[width=0.75\textwidth,height=\textheight]{images/descriptives12.png}

}

\caption{\label{fig-descriptives12}Split the variable HDRS by treatment
group}

\end{figure}%

We can now select the relevant descriptive statistics such as
\texttt{Percantiles}, \texttt{Skewness}, \texttt{Kurtosis} and the
\texttt{Shapiro-Wilk} test from the \texttt{Statistics} section:

\begin{figure}

\centering{

\includegraphics[width=0.75\textwidth,height=\textheight]{images/descriptives13.png}

}

\caption{\label{fig-descriptives13}In the Statistics section select the
descriptive statistics of interest.}

\end{figure}%

Once we have selected our descriptive statistics, a table will appear in
the output window on our right-hand side, as shown below:

\begin{figure}

\centering{

\includegraphics{images/descriptives14.png}

}

\caption{\label{fig-descriptives14}Descriptive statistics of HDSR by
treatment group}

\end{figure}%

The means are close to medians (20.3 vs 21 and 21.5 vs 21). The skewness
is approximately zero (symmetric distribution) and the (excess) kurtosis
is close to zero (mesokurtic distribution) indicating normal
distributions for both groups.

Additionally, the Shapiro-Wilk tests of normality suggest that the data
for the \texttt{HDRS} in both groups, paroxetine and placebo, are
normally distributed (p=0.67 \textgreater0.05 and p=0.61
\textgreater0.05, respectively). (NOTE: If the \(p \geq 0.05\), then the
data came from a normally distributed population).

\begin{tcolorbox}[enhanced jigsaw, bottomrule=.15mm, title=\textcolor{quarto-callout-important-color}{\faExclamation}\hspace{0.5em}{Remember: Hypothesis testing for Shapiro-Wilk test for normality}, toprule=.15mm, opacitybacktitle=0.6, coltitle=black, opacityback=0, rightrule=.15mm, leftrule=.75mm, colback=white, bottomtitle=1mm, toptitle=1mm, colframe=quarto-callout-important-color-frame, arc=.35mm, breakable, titlerule=0mm, left=2mm, colbacktitle=quarto-callout-important-color!10!white]

\(H_{0}\): the data came from a normally distributed population.

\(H_{1}\): the data tested are not normally distributed.

\begin{itemize}
\tightlist
\item
  If p − value \textless{} 0.05, reject the null hypothesis, \(H_{0}\).
\item
  If p − value ≥ 0.05, do not reject the null hypothesis, \(H_{0}\).
\end{itemize}

\end{tcolorbox}

Then we can check the \texttt{Density} from \texttt{Histograms} in the
\texttt{Plot} section, as shown below (Figure~\ref{fig-descriptives15}):

\begin{figure}

\centering{

\includegraphics[width=0.75\textwidth,height=\textheight]{images/plots_density.png}

}

\caption{\label{fig-plots_density}In the Plot section select Density
from Histograms.}

\end{figure}%

A graph is generated in the output window on our right-hand side, as
shown below:

\begin{figure}

\centering{

\includegraphics[width=0.75\textwidth,height=\textheight]{images/descriptives15.png}

}

\caption{\label{fig-descriptives15}In the Plots section select Density
from Histograms.}

\end{figure}%

The above figure shows that the data are close to symmetry and the
assumption of a normal distribution is reasonable.

\textbf{B. Homogeneity of variance}

The second assumption that should be satisfied is the homogeneity of
variance. We observe in the summary table of
Figure~\ref{fig-descriptives14} that the two standard deviations (3.65
vs 3.41) are similar (see also below the Levene's test for equality of
variances in Figure~\ref{fig-t_test4}).

~

\subsection{Run the Student's t-test}\label{run-the-students-t-test}

\begin{tcolorbox}[enhanced jigsaw, bottomrule=.15mm, title={Perform a Student's t-test}, toprule=.15mm, opacitybacktitle=0.6, coltitle=black, opacityback=0, rightrule=.15mm, leftrule=.75mm, colback=white, bottomtitle=1mm, toptitle=1mm, colframe=quarto-callout-note-color-frame, arc=.35mm, breakable, titlerule=0mm, left=2mm, colbacktitle=quarto-callout-note-color!10!white]

We will perform a Student's t-test to test the null hypothesis that the
mean HDRS score is the same for both groups (paroxetine and placebo).

We select:

\includegraphics[width=4.9in,height=0.52in]{lab6_files/figure-latex/mermaid-figure-4.png}

\begin{figure}[H]

\centering{

\includegraphics[width=0.75\textwidth,height=\textheight]{images/t_test1.png}

}

\caption{\label{fig-t_test1}Conducting an Independent Samples T-Test.}

\end{figure}%

The \texttt{Independent\ Samples\ T-Test} dialogue box opens. Drag and
drop the numeric variable \texttt{HSDR} to Dependent Variables and the
independent variable \texttt{treatment} to Grouping Variable, as shown
below Figure~\ref{fig-t_test2}:

\begin{figure}[H]

\centering{

\includegraphics[width=0.75\textwidth,height=\textheight]{images/t_test2.png}

}

\caption{\label{fig-t_test2}The Independent Samples T-Test dialogue box}

\end{figure}%

We observe that we can select between the following three Tests:
Students's (the default), Welch's, or Mann-Whitney U. At the moment, we
keep the default choice of Students's test. From
\texttt{Additional\ Statistics} check the \texttt{Mean\ difference},
\texttt{Confidence\ Intervals}, \texttt{Descriptive}, and
\texttt{Descriptive\ plots} boxes. Finally, from
\texttt{Assumption\ Checks} tick the \texttt{Homogeneity\ test}. We will
end up with the following screen:

\begin{figure}[H]

\centering{

\includegraphics[width=0.75\textwidth,height=\textheight]{images/t_test3.png}

}

\caption{\label{fig-t_test3}Additional statistics and tests.}

\end{figure}%

First, we look at the table of \texttt{Levene\textquotesingle{}s\ test}
for equality of variances (Figure~\ref{fig-t_test4}):

\begin{figure}[H]

\centering{

\includegraphics[width=0.6\textwidth,height=\textheight]{images/t_test4.png}

}

\caption{\label{fig-t_test4}Levene's test.}

\end{figure}%

\begin{tcolorbox}[enhanced jigsaw, bottomrule=.15mm, title=\textcolor{quarto-callout-important-color}{\faExclamation}\hspace{0.5em}{Remember: Hypothesis testing for Levene's test for equality of variances}, toprule=.15mm, opacitybacktitle=0.6, coltitle=black, opacityback=0, rightrule=.15mm, leftrule=.75mm, colback=white, bottomtitle=1mm, toptitle=1mm, colframe=quarto-callout-important-color-frame, arc=.35mm, breakable, titlerule=0mm, left=2mm, colbacktitle=quarto-callout-important-color!10!white]

\(H_{0}\): the variances of HDRs in two groups are equal

\(H_{1}\): the variances of HDRs in two groups are not equal

\begin{itemize}
\tightlist
\item
  If p − value \textless{} 0.05, reject the null hypothesis, \(H_{0}\).
\item
  If p − value ≥ 0.05, do not reject the null hypothesis, \(H_{0}\).
\end{itemize}

\end{tcolorbox}

Since p = 0.646 \textgreater{} 0.05, the \(H_0\) of the Levene's test is
not rejected and we keep the default choice of \textbf{Students's} test
(Figure~\ref{fig-t_test3}). (NOTE: If the \(p \geq 0.05\), then the
population variances of HDRS in two groups groups are assumed equal).

If the assumption of equal variances is not satisfied (Levene's test
gives p \textless{} 0.05, reject \(H_0\)), the Welch's test should be
used from the available Tests in Jamovi (Figure~\ref{fig-t_test3}).

Next, we can inspect again the results in the group descriptives table
(Figure~\ref{fig-t_test5}) and pertinent plots
(Figure~\ref{fig-t_test6}):

\begin{figure}[H]

\centering{

\includegraphics[width=0.75\textwidth,height=\textheight]{images/t_test5.png}

}

\caption{\label{fig-t_test5}Group descriptives.}

\end{figure}%

\begin{figure}[H]

\centering{

\includegraphics[width=0.75\textwidth,height=\textheight]{images/t_test6.png}

}

\caption{\label{fig-t_test6}Plot of mean (95\% CI) and median of HDRS by
treatment.}

\end{figure}%

Finally, we present the results of the Student's t-test in the table of
the Figure~\ref{fig-t_test7}:

\begin{figure}[H]

\centering{

\includegraphics[width=0.9\textwidth,height=\textheight]{images/t_test7.png}

}

\caption{\label{fig-t_test7}The results of the Student's t-test.}

\end{figure}%

The p-value = 0.16 is greater than 0.05. There is \textbf{no evidence}
of a significant difference in mean HDRS between the two groups (failed
to reject \(H_0\)). The difference between means (20.33 - 21.49) equals
to -1.16 units of the HDRS and note that the 95\% confidence interval of
the difference in means (-2.78 to 0.47) \textbf{includes} the
hypothesized null value of 0. Based on these results, there is not
evidence that paroxetine is effective as a treatment for bipolar
depression.

Note that the paroxetine sample (n= 33) has 32 (33-1) degrees of freedom
and the placebo sample (n= 43) has 42 (43-1), so we have 74 (32 + 42) df
in total. Another way of thinking of this is that the complete sample
size is 76, and we have estimated two parameters from the data (the two
means), so we have 76-2 = 74 df.

The Student t-test for two independent samples does not have any
restrictions on \(n_1\) and \(n_2\) ---\textbf{they can be equal or
unequal}. However, equal samples are preferred because when a total of
2n subjects are available, their equal division among the groups
maximizes the power to detect a specified difference.

\end{tcolorbox}

\begin{tcolorbox}[enhanced jigsaw, bottomrule=.15mm, title={Mann-Whitney U test}, toprule=.15mm, opacitybacktitle=0.6, coltitle=black, opacityback=0, rightrule=.15mm, leftrule=.75mm, colback=white, bottomtitle=1mm, toptitle=1mm, colframe=quarto-callout-important-color-frame, arc=.35mm, breakable, titlerule=0mm, left=2mm, colbacktitle=quarto-callout-important-color!10!white]

When there is violation of normality, the Mann-Whitney U test can be
selected from the available Tests in Jamovi (Figure~\ref{fig-t_test4}).
This test compares two independent samples based on the ranks of the
values and is often considered the non-parametric equivalent to the
Student's t-test.

\end{tcolorbox}

~

\section{Paired samples t-test}\label{paired-samples-t-test}

The paired samples design can effectively reduce the effect of
non-treatment factors and improve the efficiency of the experiment. A
paired samples t-test is used to estimate whether the means of two
related measurements are significantly different from one another.

Open the dataset named \texttt{weight} from the file tab in the menu:

\begin{figure}

\centering{

\includegraphics[width=0.85\textwidth,height=\textheight]{images/weight_data.png}

}

\caption{\label{fig-weight_data}The weight dataset}

\end{figure}%

The dataset \texttt{weight} contains the birth and discharge weight of
25 newborns (Figure~\ref{fig-weight_data}). Double-click on the name of
the variables \texttt{birth\_weight} and \texttt{discharge\_weight} to
change the measure type from nominal
\includegraphics[width=0.23958in,height=0.21875in]{images/nominal.png}
to continuous
\includegraphics[width=0.23958in,height=0.21875in]{images/continuous.png}.

\subsection{Research question}\label{research-question-1}

We might ask if the \textbf{mean difference} of the weight in birth and
in discharge equals to zero or not. If the \textbf{differences} between
the pairs of measurements are normally distributed, a paired t-test is
the most appropriate statistical test.

\subsection{Hypothesis Testsing for the paired samples
t-test}\label{hypothesis-testsing-for-the-paired-samples-t-test}

\begin{tcolorbox}[enhanced jigsaw, bottomrule=.15mm, title={Null hypothesis and alternative hypothesis}, toprule=.15mm, opacitybacktitle=0.6, coltitle=black, opacityback=0, rightrule=.15mm, leftrule=.75mm, colback=white, bottomtitle=1mm, toptitle=1mm, colframe=quarto-callout-note-color-frame, arc=.35mm, breakable, titlerule=0mm, left=2mm, colbacktitle=quarto-callout-note-color!10!white]

\begin{itemize}
\tightlist
\item
  H\textsubscript{0}: the mean difference in weight is zero
  (\(\mu_{d} = 0\))
\item
  H\textsubscript{1}: the mean difference in weight is non-zero
  (\(\mu_{d} \neq 0\))
\end{itemize}

\end{tcolorbox}

\subsection{Assumptions}\label{assumptions-1}

\begin{tcolorbox}[enhanced jigsaw, bottomrule=.15mm, title={Check if the following assumption is satisfied}, toprule=.15mm, opacitybacktitle=0.6, coltitle=black, opacityback=0, rightrule=.15mm, leftrule=.75mm, colback=white, bottomtitle=1mm, toptitle=1mm, colframe=quarto-callout-note-color-frame, arc=.35mm, breakable, titlerule=0mm, left=2mm, colbacktitle=quarto-callout-note-color!10!white]

\begin{enumerate}
\def\labelenumi{\arabic{enumi}.}
\tightlist
\item
  The \textbf{differences} between the pairs of measurements,
  \(d_{i}\)s, are normally distributed. (NOTE: It is not essential for
  the original observations to be normally distributed).
\end{enumerate}

\end{tcolorbox}

\textbf{Explore the characteristics of the distribution of differences,}
\(d_{i}\)

First, we have to calculate the differences
\(d_{i}= birth\_weight_i - discharge\_weight_i\)
(Figure~\ref{fig-paired2}) from Data tab in the main menu of Jamovi. For
more details go to the section \emph{11.6 Transforming data: Computing a
new variable} in Chapter~\ref{sec-lab1}.

\begin{figure}

\centering{

\includegraphics[width=0.9\textwidth,height=\textheight]{images/paired2.png}

}

\caption{\label{fig-paired2}Calculation of the variable of differences
d}

\end{figure}%

The distributions of the differences,\(d_{i}\), can be explored with
appropriate plots and summary statistics.

On the Jamovi top menu navigate to

\includegraphics[width=4.06in,height=0.52in]{lab6_files/figure-latex/mermaid-figure-3.png}

as shown below in Figure~\ref{fig-weight0}.

\begin{figure}

\centering{

\includegraphics[width=0.75\textwidth,height=\textheight]{images/weight0.png}

}

\caption{\label{fig-weight0}In the Analyses Tab select Exploration and
click on Descriptives.}

\end{figure}%

The \texttt{Descriptives} dialogue box opens. Drag the variable
\texttt{d} into the \texttt{Variables} box, as shown below
(Figure~\ref{fig-paired3}):

\begin{figure}

\centering{

\includegraphics[width=0.75\textwidth,height=\textheight]{images/paired3.png}

}

\caption{\label{fig-paired3}Drag the variable of the differences
\texttt{d} into the \texttt{Variables} box}

\end{figure}%

We can now select the relevant descriptive statistics such as
\texttt{Percantiles}, \texttt{Skewness}, \texttt{Kurtosis} and the
\texttt{Shapiro-Wilk} test from the \texttt{Statistics} section:

\begin{figure}[H]

{\centering \includegraphics[width=0.75\textwidth,height=\textheight]{images/descriptives13.png}

}

\caption{In the Statistics section select the descriptive statistics of
interest.}

\end{figure}%

Once we have selected our descriptive statistics, a table will appear in
the output window on our right-hand side, as shown below:

\begin{figure}

\centering{

\includegraphics{images/paired4.png}

}

\caption{\label{fig-paired4}Descriptive statistics of the differences.}

\end{figure}%

The mean is close to median (39.6 vs 40). Moreover, both skewness and
(excess) kurtosis are approximately zero indicating a symmetric and
mesokurtic distribution of the weight differences.

Then we can check the \texttt{Density} from \texttt{Histograms} in the
\texttt{Plot} section, as shown below (Figure~\ref{fig-plots_density}):

\begin{figure}[H]

{\centering \includegraphics[width=0.75\textwidth,height=\textheight]{images/plots_density.png}

}

\caption{In the Plot section select Density from Histograms.}

\end{figure}%

A graph is generated in the output window on our right-hand side, as
shown below:

\begin{figure}

\centering{

\includegraphics[width=0.65\textwidth,height=\textheight]{images/paired5.png}

}

\caption{\label{fig-paired5}In the Plots section select Density from
Histograms.}

\end{figure}%

The above figure shows that the data are close to symmetry and the
assumption of a normal distribution is reasonable.

Additionally, the Shapiro-Wilk test of normality suggests that the data
for the differences, \(d_{i}\), are normally distributed (p=0.74
\textgreater0.05). (NOTE: If the \(p \geq 0.05\), then the data came
from a normally distributed population).

\subsection{Run the paired samples
t-test}\label{run-the-paired-samples-t-test}

\begin{tcolorbox}[enhanced jigsaw, bottomrule=.15mm, title={Perform a paired samples t-test}, toprule=.15mm, opacitybacktitle=0.6, coltitle=black, opacityback=0, rightrule=.15mm, leftrule=.75mm, colback=white, bottomtitle=1mm, toptitle=1mm, colframe=quarto-callout-note-color-frame, arc=.35mm, breakable, titlerule=0mm, left=2mm, colbacktitle=quarto-callout-note-color!10!white]

We will perform a paired samples t-test to test the null hypothesis that
the mean difference in weight is zero.

We select:

\includegraphics[width=4.44in,height=0.52in]{lab6_files/figure-latex/mermaid-figure-2.png}

\begin{figure}[H]

\centering{

\includegraphics[width=0.75\textwidth,height=\textheight]{images/paired1.png}

}

\caption{\label{fig-paired1}Conducting a Paired Samples T-Test.}

\end{figure}%

The \texttt{Paired\ Samples\ T-Test} dialogue box opens. Drag and drop
the variables \texttt{birth\_weight} and \texttt{discharge\_weight} to
Paired Variables, as shown below Figure~\ref{fig-paired6}:

\begin{figure}[H]

\centering{

\includegraphics[width=0.75\textwidth,height=\textheight]{images/paired6.png}

}

\caption{\label{fig-paired6}The Paired Samples T-Test dialogue box}

\end{figure}%

We observe that we can select between the following two Tests:
Students's or Wilcoxon rank. We keep the default choice of Students's
paired t-test. Moreover, from \texttt{Additional\ Statistics} check the
\texttt{Mean\ difference}, \texttt{Confidence\ Intervals},
\texttt{Descriptive}, and \texttt{Descriptive\ plots} boxes. Finally,
from \texttt{Assumption\ Checks} tick the \texttt{Normality\ test}. We
will end up with the following screen:

\begin{figure}[H]

\centering{

\includegraphics[width=0.75\textwidth,height=\textheight]{images/paired7.png}

}

\caption{\label{fig-paired7}Additional statistics and tests.}

\end{figure}%

Next, we can inspect the results in the table with descriptive
statistics (Figure~\ref{fig-paired8}) and plots
(Figure~\ref{fig-t_test6}):

\begin{figure}[H]

\centering{

\includegraphics[width=0.75\textwidth,height=\textheight]{images/paired8.png}

}

\caption{\label{fig-paired8}Table with descriptive statistics.}

\end{figure}%

\begin{figure}[H]

\centering{

\includegraphics[width=0.75\textwidth,height=\textheight]{images/paired9.png}

}

\caption{\label{fig-paired9}Plot of mean and median of \emph{birth
weigt} and \emph{discharge\_weight}.}

\end{figure}%

The Shapiro-Wilk test of normality of the differences has previously
calculated (Figure~\ref{fig-paired4}) and is also presented below:

\begin{figure}[H]

\centering{

\includegraphics[width=0.55\textwidth,height=\textheight]{images/paired90.png}

}

\caption{\label{fig-paired90}Test of normality of the differences.}

\end{figure}%

Finally, we present the results of the Student's paired samples t-test
in the table of the Figure~\ref{fig-paired10}:

\begin{figure}[H]

\centering{

\includegraphics[width=0.95\textwidth,height=\textheight]{images/paired10.png}

}

\caption{\label{fig-paired10}The results of the Paired Samples t-test.}

\end{figure}%

There was a \textbf{significant} reduction in weight (39.6 g) after the
discharge (p-value \textless0.001 that is lower than 0.05; reject
\(H_0\)). Note that the 95\% confidence interval (26.3 to 52.9)
\textbf{doesn't include} the null hypothesized value of 0. However, is
this reduction of clinical importance?

\end{tcolorbox}

\begin{tcolorbox}[enhanced jigsaw, bottomrule=.15mm, title={Wilcoxon Signed-Rank test}, toprule=.15mm, opacitybacktitle=0.6, coltitle=black, opacityback=0, rightrule=.15mm, leftrule=.75mm, colback=white, bottomtitle=1mm, toptitle=1mm, colframe=quarto-callout-important-color-frame, arc=.35mm, breakable, titlerule=0mm, left=2mm, colbacktitle=quarto-callout-important-color!10!white]

When there is violation of normality, the Wilcoxon Rank test can be
selected from the available Tests in Jamovi (Figure~\ref{fig-paired7}).
This test is based on the sign and the magnitude of the rank of the
differences between pairs of measurements, rather than the actual
values. It is often considered the non-parametric equivalent to the
Student's paired samples t-test.

\end{tcolorbox}

\chapter{LAB VII: Inference for numerical data (\textgreater2
samples)}\label{sec-lab7}

When we have finished this Lab, we should be able to:

\begin{tcolorbox}[enhanced jigsaw, bottomrule=.15mm, title={Learning objectives}, toprule=.15mm, opacitybacktitle=0.6, coltitle=black, opacityback=0, rightrule=.15mm, leftrule=.75mm, colback=white, bottomtitle=1mm, toptitle=1mm, colframe=quarto-callout-caution-color-frame, arc=.35mm, breakable, titlerule=0mm, left=2mm, colbacktitle=quarto-callout-caution-color!10!white]

\begin{itemize}
\tightlist
\item
  Applying hypothesis testing
\item
  Compare more than two independent samples
\item
  Interpret the results
\end{itemize}

\end{tcolorbox}

\section{Introduction}\label{introduction}

The one-way analysis of variance (one-way ANOVA) or the non-parametric
Kruskal-Wallis test are used to detect whether there are any differences
between more than two independent (unrelated) samples.

Although, these tests can detect a difference between several groups
they do not inform about which groups are different from the others. At
first sight we might clarify the question by comparing all groups in
pairs with t-tests or Mann-Whitney U tests. However, that procedure may
lead us to the wrong conclusions (known as multiple comparisons
problem).

Why is this procedure inappropriate? Quite simply, because we would be
wrongly testing the null hypothesis. Each comparison one conducts
increases the likelihood of committing at least one Type I error within
a set of comparisons (famillywise Type I error rate).

This is the reason why, after an ANOVA or Kruskal-Wallis test concluding
on a difference between groups, we should not just compare all possible
pairs of groups with t-tests or Mann-Whitney U tests. Instead we perform
statistical tests that take into account the number of comparisons (post
hoc tests). Some of the more commonly used ones are Tukey test,
Games-Howell test, and Bonferroni correction.

\section{One-way Analysis of Variance
(ANOVA)}\label{one-way-analysis-of-variance-anova}

One-way analysis of variance, usually referred to as one-way ANOVA, is a
statistical test used when we want to compare several means. We may
think of it as an extension of Student's t-test to the case of more than
two samples.

\subsection{Opening the file}\label{opening-the-file-2}

Open the dataset named ``dataDWL'' from the file tab in the menu:

\begin{figure}

\centering{

\includegraphics[width=0.85\textwidth,height=\textheight]{images/dataDWL.png}

}

\caption{\label{fig-dataDWL}The dataDWL dataset}

\end{figure}%

The dataset ``dataDWL'' has 60 participants and includes two variables
(Figure~\ref{fig-dataDWL}). The numeric \texttt{WeightLoss} variable and
the \texttt{Diet} variable (with levels \texttt{A}, \texttt{B},
\texttt{C} and \texttt{D}).

\subsection{Research question}\label{research-question-2}

Consider the example of the variations between weight loss according to
four different types of diet (\texttt{A}, \texttt{B}, \texttt{C}, and
\texttt{D}). The question that may be asked is: does the average weight
loss (units in kg) differ according to the diet?

\subsection{Hypothesis Testsing for the ANOVA
test}\label{hypothesis-testsing-for-the-anova-test}

\begin{tcolorbox}[enhanced jigsaw, bottomrule=.15mm, title={Null hypothesis and alternative hypothesis}, toprule=.15mm, opacitybacktitle=0.6, coltitle=black, opacityback=0, rightrule=.15mm, leftrule=.75mm, colback=white, bottomtitle=1mm, toptitle=1mm, colframe=quarto-callout-note-color-frame, arc=.35mm, breakable, titlerule=0mm, left=2mm, colbacktitle=quarto-callout-note-color!10!white]

\begin{itemize}
\tightlist
\item
  \(H_0\): all group means are equal (the means of weight loss in the
  four diets are equal: \(\mu_{A} = \mu_{B} = \mu_{C} = \mu_{D}\))
\item
  \(H_1\): at least one group mean differs from the others (there is at
  least one diet with mean weight loss different from the others)
\end{itemize}

\end{tcolorbox}

\subsection{Assumptions}\label{assumptions-2}

\begin{tcolorbox}[enhanced jigsaw, bottomrule=.15mm, title={Check if the following assumptions are satisfied}, toprule=.15mm, opacitybacktitle=0.6, coltitle=black, opacityback=0, rightrule=.15mm, leftrule=.75mm, colback=white, bottomtitle=1mm, toptitle=1mm, colframe=quarto-callout-note-color-frame, arc=.35mm, breakable, titlerule=0mm, left=2mm, colbacktitle=quarto-callout-note-color!10!white]

\begin{enumerate}
\def\labelenumi{\arabic{enumi}.}
\tightlist
\item
  The dependent variable, \texttt{WeightLoss}, should be approximately
  \textbf{normally} distributed for all groups
\item
  The data in groups have similar \textbf{variance} (also named as
  homogeneity of variance or homoscedasticity)
\end{enumerate}

\end{tcolorbox}

\textbf{A. Explore the descriptive characteristics of distribution for
each group and check for normality}

The distributions can be explored visually with appropriate plots.
Additionally, summary statistics and significance tests to check for
normality (e.g., Shapiro-Wilk test) and for equality of variances (e.g.,
Levene's test) can be used.

On the Jamovi top menu navigate to

\includegraphics[width=4.06in,height=0.52in]{lab7_files/figure-latex/mermaid-figure-1.png}

as shown below in Figure~\ref{fig-diet0}.

\begin{figure}

\centering{

\includegraphics[width=0.75\textwidth,height=\textheight]{images/diet0.png}

}

\caption{\label{fig-diet0}In the menu at the top, choose Analyses
\textgreater{} Exploration
\includegraphics[width=0.36458in,height=0.34375in]{images/explor_icon.png}
\textgreater{} Descriptives.}

\end{figure}%

The \emph{Descriptives dialogue box} opens. Drag the variable
\texttt{WeightLoss} into the \emph{Variables} field and \emph{split} it
by the \texttt{Diet} variable. Additionally, select \textbf{Variable
across rows}, as shown below (Figure~\ref{fig-diet1}):

\begin{figure}

\centering{

\includegraphics[width=0.75\textwidth,height=\textheight]{images/diet1.png}

}

\caption{\label{fig-diet1}Split the variable \textbf{WeightLoss} by
\textbf{Diet} group and select Variables across rows.}

\end{figure}%

We can now select the relevant descriptive statistics such as
\textbf{Percantiles}, \textbf{Skewness}, \textbf{Kurtosis} and the
\textbf{Shapiro-Wilk} test from the \emph{Statistics} section
(Figure~\ref{fig-descriptives13}):

\begin{figure}

\centering{

\includegraphics[width=0.7\textwidth,height=\textheight]{images/descriptives13.png}

}

\caption{\label{fig-descriptives13}In the Statistics section select the
descriptive statistics of interest.}

\end{figure}%

Once we have selected our descriptive statistics, a table will appear in
the output window on our right-hand side, as shown below
(Figure~\ref{fig-diet2}):

\begin{figure}

\centering{

\includegraphics{images/diet2.png}

}

\caption{\label{fig-diet2}Descriptive statistics of \textbf{WeightLoss}
by \textbf{Diet} group (click on figure to zoom in).}

\end{figure}%

The means are close to medians and the standard deviations are also
similar indicating normal distributions for all groups. Additionally,
both shape measures, skewness and (excess) kurtosis, have values in the
acceptable range {[}-1, 1{]} which indicate symmetric and mesokurtic
distributions, respectively.

The Shapiro-Wilk tests of normality suggest that the data for the
\texttt{WeightLoss} in all groups are normally distributed (p
\textgreater{} 0.05 \(\Rightarrow H_0\) is not rejected).

\begin{tcolorbox}[enhanced jigsaw, bottomrule=.15mm, title=\textcolor{quarto-callout-important-color}{\faExclamation}\hspace{0.5em}{Remember: Hypothesis testing for Shapiro-Wilk test for normality}, toprule=.15mm, opacitybacktitle=0.6, coltitle=black, opacityback=0, rightrule=.15mm, leftrule=.75mm, colback=white, bottomtitle=1mm, toptitle=1mm, colframe=quarto-callout-important-color-frame, arc=.35mm, breakable, titlerule=0mm, left=2mm, colbacktitle=quarto-callout-important-color!10!white]

\(H_{0}\): the data came from a normally distributed population.

\(H_{1}\): the data tested are not normally distributed.

\begin{itemize}
\tightlist
\item
  If p − value \textless{} 0.05, reject the null hypothesis, \(H_{0}\).
\item
  If p − value ≥ 0.05, do not reject the null hypothesis, \(H_{0}\).
\end{itemize}

\end{tcolorbox}

Then we can check the \textbf{Density} box from \emph{Histograms} in the
\emph{Plot} section, as shown below (Figure~\ref{fig-plots_density}):

\begin{figure}

\centering{

\includegraphics[width=0.75\textwidth,height=\textheight]{images/plots_density.png}

}

\caption{\label{fig-plots_density}In the Plots section select Density
from Histograms.}

\end{figure}%

A graph (Figure~\ref{fig-diet3}) is generated in the output window on
our right-hand side:

\begin{figure}

\centering{

\includegraphics[width=0.7\textwidth,height=\textheight]{images/diet3.png}

}

\caption{\label{fig-diet3}The Density plot of \textbf{WeightLoss} for
each \textbf{Diet}.}

\end{figure}%

The above density plots show that the data are close to symmetry and the
assumption of a normal distribution is reasonable for all diet groups.

\textbf{B. Homogeneity of variance}

The second assumption that should be satisfied is the homogeneity of
variance. We observe in the summary table of Figure~\ref{fig-diet2} that
the standard deviations are similar (see also below the Levene's test
for equality of variances in Figure~\ref{fig-t_test4}).

~

\subsection{Run the ANOVA test}\label{run-the-anova-test}

\begin{tcolorbox}[enhanced jigsaw, bottomrule=.15mm, title={Perform ANOVA in Jamovi}, toprule=.15mm, opacitybacktitle=0.6, coltitle=black, opacityback=0, rightrule=.15mm, leftrule=.75mm, colback=white, bottomtitle=1mm, toptitle=1mm, colframe=quarto-callout-note-color-frame, arc=.35mm, breakable, titlerule=0mm, left=2mm, colbacktitle=quarto-callout-note-color!10!white]

We will perform ANOVA to test the null hypothesis that the mean
\texttt{WeightLoss} is the same for all \texttt{Diet} groups.

On the Jamovi top menu navigate to

\includegraphics[width=4in,height=0.52in]{lab7_files/figure-latex/mermaid-figure-2.png}

as shown below in Figure~\ref{fig-diet4}.

\begin{figure}[H]

\centering{

\includegraphics[width=0.75\textwidth,height=\textheight]{images/diet4.png}

}

\caption{\label{fig-diet4}Conducting ANOVA test in Jamovi. In the menu
at the top, choose Analyses \textgreater{} ANOVA
\includegraphics[width=0.36458in,height=0.34375in]{images/anova_icon.png}
\textgreater{} One-Way ANOVA.}

\end{figure}%

The \emph{One-Way ANOVA dialogue box} opens. Drag and drop the
\texttt{WeightLoss} to Dependent Variables field and the \texttt{Diet}
to Grouping Variable, as shown below Figure~\ref{fig-diet5}:

\begin{figure}[H]

\centering{

\includegraphics[width=0.75\textwidth,height=\textheight]{images/diet5.png}

}

\caption{\label{fig-diet5}One-Way ANOVA dialogue box. Drag the
\textbf{WeightLoss} into the Dependent Variables field and the
\textbf{Diet} into the Grouping Variable field.}

\end{figure}%

We observe that we can select between the following two Tests: Welch's
test (the default), or Fisher's test. At the moment, we keep the default
choice. Moreover, from \textbf{Additional Statistics} check the
\textbf{Descriptive} and \textbf{Descriptive plots} boxes. Finally, from
\textbf{Assumption Checks} tick the \textbf{Homogeneity test} box. We
will end up with the following screen:

\begin{figure}[H]

\centering{

\includegraphics[width=0.75\textwidth,height=\textheight]{images/diet6.png}

}

\caption{\label{fig-diet6}Additional statistics and tests.}

\end{figure}%

\end{tcolorbox}

First, we look at the table of \textbf{Levene's test} for equality of
variances (Figure~\ref{fig-diet7}):

\begin{figure}

\centering{

\includegraphics[width=0.6\textwidth,height=\textheight]{images/diet7.png}

}

\caption{\label{fig-diet7}Levene's test.}

\end{figure}%

\begin{tcolorbox}[enhanced jigsaw, bottomrule=.15mm, title=\textcolor{quarto-callout-important-color}{\faExclamation}\hspace{0.5em}{Remember: Hypothesis testing for Levene's test for equality of variances}, toprule=.15mm, opacitybacktitle=0.6, coltitle=black, opacityback=0, rightrule=.15mm, leftrule=.75mm, colback=white, bottomtitle=1mm, toptitle=1mm, colframe=quarto-callout-important-color-frame, arc=.35mm, breakable, titlerule=0mm, left=2mm, colbacktitle=quarto-callout-important-color!10!white]

\(H_{0}\): the variances of \texttt{WeightLoss} in all groups are equal
(\(σ^2_A=σ^2_B=σ^2_C=σ^2_D\))

\(H_{1}\): the variances of \texttt{WeightLoss} differ between groups
(\(σ^2_i\neq σ^2_j\), where \(i,j= A, B, C, D\) and \(i\neq j\))

\begin{itemize}
\tightlist
\item
  If p − value \textless{} 0.05, reject the null hypothesis, \(H_{0}\).
\item
  If p − value ≥ 0.05, do not reject the null hypothesis, \(H_{0}\).
\end{itemize}

\end{tcolorbox}

Since p = 0.583 \textgreater{} 0.05, the \(H_0\) of the Levene's test is
not rejected and we have to perform the \textbf{Fisher's test} which
assumes equal variances (Figure~\ref{fig-diet8}). So, let's tick on the
\textbf{Assume equal (Fisher's)} box. (NOTE: If the \(p \geq 0.05\),
then the population variances of \texttt{WeightLoss} in all groups are
assumed equal).

\begin{figure}

\centering{

\includegraphics[width=0.75\textwidth,height=\textheight]{images/diet8.png}

}

\caption{\label{fig-diet8}Fisher's ANOVA test.}

\end{figure}%

\begin{tcolorbox}[enhanced jigsaw, bottomrule=.15mm, title={ANOVA test (Welch's option)}, toprule=.15mm, opacitybacktitle=0.6, coltitle=black, opacityback=0, rightrule=.15mm, leftrule=.75mm, colback=white, bottomtitle=1mm, toptitle=1mm, colframe=quarto-callout-important-color-frame, arc=.35mm, breakable, titlerule=0mm, left=2mm, colbacktitle=quarto-callout-important-color!10!white]

If the assumption of equal variances is not satisfied (Levene's test
gives p \textless{} 0.05, reject \(H_0\)), the Welch's option of ANOVA
should be used from the available Tests in Jamovi
(Figure~\ref{fig-diet8}).

\end{tcolorbox}

Next, we can inspect again the results in the group descriptives table
(Figure~\ref{fig-diet9}) and pertinent plots (Figure~\ref{fig-diet9}):

\begin{figure}

\centering{

\includegraphics[width=0.65\textwidth,height=\textheight]{images/diet9.png}

}

\caption{\label{fig-diet9}Group descriptive statistics.}

\end{figure}%

\begin{figure}

\centering{

\includegraphics[width=0.55\textwidth,height=\textheight]{images/diet10.png}

}

\caption{\label{fig-diet10}Plot of mean (95\% CI) of WeightLoss by
Diet.}

\end{figure}%

From the Figure~\ref{fig-diet10} we observe that the participants
following the diet C have on average the higher weight loss.

Finally, we present the results of the Fisher's ANOVA test in the table
of Figure~\ref{fig-diet11}:

\begin{figure}

\centering{

\includegraphics[width=0.55\textwidth,height=\textheight]{images/diet11.png}

}

\caption{\label{fig-diet11}The results of the Fisher's ANOVA test.}

\end{figure}%

In Figure~\ref{fig-diet11}, F= 6.12 indicates the F-statistic:

\[F= \frac{variation \ between \ sample \ means}{variation \ within \ the \ samples}\]

Note that we compare this value to an F-distribution (F-test). The
degrees of freedom in the numerator (df1) and the denominator (df2) are
3 and 56, respectively.

The p-value=0.001 is less than 0.05 (reject \(H_0\) of the ANOVA test).
\textbf{There is at least one diet with mean weight loss which is
different from the other means}.

~

\subsection{Run post-hoc tests}\label{run-post-hoc-tests}

\begin{tcolorbox}[enhanced jigsaw, bottomrule=.15mm, title={Perform post-hoc tests}, toprule=.15mm, opacitybacktitle=0.6, coltitle=black, opacityback=0, rightrule=.15mm, leftrule=.75mm, colback=white, bottomtitle=1mm, toptitle=1mm, colframe=quarto-callout-note-color-frame, arc=.35mm, breakable, titlerule=0mm, left=2mm, colbacktitle=quarto-callout-note-color!10!white]

A \textbf{significant} one-way ANOVA is generally followed by post-hoc
tests to perform multiple pairwise comparisons between groups. From the
\emph{One-Way ANOVA dialogue box} click on \emph{Post-Hoc Tests}
section. We have got the following two options:

\begin{itemize}
\item
  Games-Howell (unequal variances)
\item
  Tukey (equal variances)
\end{itemize}

Based on the result of Levene's test (p = 0.583 \textgreater{} 0.05, the
\(H_0\) is not rejected) (Figure~\ref{fig-diet7}), we should select the
\textbf{Tukey (equal variances)} post-hoc test. Additionally, check the
\textbf{Flag significant comparisons} as shown below
(Figure~\ref{fig-diet12}):

\begin{figure}[H]

\centering{

\includegraphics[width=0.7\textwidth,height=\textheight]{images/diet12.png}

}

\caption{\label{fig-diet12}Select the appropriate post-hoc test. For
equal variances the Tukey post hoc test.}

\end{figure}%

Once we have selected our post-hoc test, a table will appear in the
output window on our right-hand side, as shown below
(Figure~\ref{fig-diet13}):

\begin{figure}[H]

\centering{

\includegraphics[width=0.7\textwidth,height=\textheight]{images/diet13.png}

}

\caption{\label{fig-diet13}Table with the results of the Tukey post-hoc
test.}

\end{figure}%

\textbf{Interpretation}

Pairwise comparisons were carried out using the method of Tukey and the
adjusted p-values were calculated. \textbf{The weight loss following
diet C is significantly larger compared to diet A (mean difference =
2.93 kg, p=0.005 \textless0.05) or diet B (mean difference = 3.21 kg,
p=0.002 \textless0.05)}.

\end{tcolorbox}

\chapter{LAB IX: Correlation}\label{sec-lab9}

When we have finished this Lab, we should be able to:

\begin{tcolorbox}[enhanced jigsaw, bottomrule=.15mm, title={Learning objectives}, toprule=.15mm, opacitybacktitle=0.6, coltitle=black, opacityback=0, rightrule=.15mm, leftrule=.75mm, colback=white, bottomtitle=1mm, toptitle=1mm, colframe=quarto-callout-caution-color-frame, arc=.35mm, breakable, titlerule=0mm, left=2mm, colbacktitle=quarto-callout-caution-color!10!white]

\begin{itemize}
\tightlist
\item
  Understand the concept of correlation of two numeric variables.
\item
  Compute Pearson's r (or Spearmans \(r_{s}\)) correlation coefficient
  between two numeric variables
\item
  Discuss the possible meaning of correlation that we observe.
\end{itemize}

\end{tcolorbox}

~

In this Lab, we will use the data from ``LungCapacity''
dataset.(\textbf{Note:} This starts by assuming we know how to get data
into Jamovi).

\subsection{Opening the file}\label{opening-the-file-3}

Open the dataset named ``LungCapacity'' from the file tab in the menu:

\begin{figure}

\centering{

\includegraphics[width=0.85\textwidth,height=\textheight]{images/lungcap.png}

}

\caption{\label{fig-lungcap1}The ``LungCapacity'' dataset.}

\end{figure}%

The dataset ``LungCapacity'' has 725 participants and includes two
variables. The numeric variables of interest are the \texttt{Age} and
the \texttt{LungCap} (Figure~\ref{fig-lungcap1}). Double-click on the
variable name \texttt{Age} and change the measure type from nominal
\includegraphics[width=0.23958in,height=0.21875in]{images/nominal.png}
to continuous
\includegraphics[width=0.23958in,height=0.21875in]{images/continuous.png}.

\subsection{Research question}\label{research-question-3}

Let's say that we want to explore the association between age (in years)
and lung capacity (in liters) for the sample of 725 participants in a
survey.

\subsection{Hypothesis Testsing}\label{hypothesis-testsing}

\begin{tcolorbox}[enhanced jigsaw, bottomrule=.15mm, title={Null hypothesis and alternative hypothesis}, toprule=.15mm, opacitybacktitle=0.6, coltitle=black, opacityback=0, rightrule=.15mm, leftrule=.75mm, colback=white, bottomtitle=1mm, toptitle=1mm, colframe=quarto-callout-note-color-frame, arc=.35mm, breakable, titlerule=0mm, left=2mm, colbacktitle=quarto-callout-note-color!10!white]

\begin{itemize}
\item
  \(H_{0}:\) there is not association between age and lung capacity
  (\(ρ = 0\)).
\item
  \(H_{1}:\) there is association between age and lung capacity
  (\(ρ \neq 0\)).
\end{itemize}

\end{tcolorbox}

\subsection{Graphical display with a scatter
plot}\label{graphical-display-with-a-scatter-plot}

A first step that is usually useful in studying the association between
two continuous variables is to prepare a \textbf{scatterplot} of the
data. The pattern made by the points plotted on the scatterplot usually
suggests the basic nature and strength of the association between two
variables.

On the Jamovi top menu navigate to

\includegraphics[width=4in,height=0.52in]{lab9_files/figure-latex/mermaid-figure-1.png}

as shown below in Figure~\ref{fig-lungcap_scatter1}.

\begin{figure}

\centering{

\includegraphics[width=0.65\textwidth,height=\textheight]{images/lungcap_scatter1.png}

}

\caption{\label{fig-lungcap_scatter1}In the menu at the top, choose
Analyses \textgreater{} Exploration
\includegraphics[width=0.36458in,height=0.34375in]{images/explor_icon.png}
\textgreater{} Scatterplot.}

\end{figure}%

The \emph{Scatterplot dialogue box} opens
(Figure~\ref{fig-lungcap_scatter3}). Transfer the \texttt{Age} and
\texttt{LungCap} variables from the left-hand pane into the
\emph{X-Axis} and \emph{Y-Axis} fields on the right-hand side,
respectively, by highlighting the variables and pressing the Arrow
Button
(\includegraphics[width=0.23958in,height=0.21875in]{images/right_arrow.png}).
Alternatively, drag and drop the variables. Finally, from
\textbf{Marginals} click on the ``\textbf{Densities}'' radio button. We
will end up with the following screen:

\begin{figure}

\centering{

\includegraphics[width=5.20833in,height=4.60417in]{images/lungcap_scatter3.png}

}

\caption{\label{fig-lungcap_scatter3}The Scatterplot dialogue box
options.}

\end{figure}%

The resulting graph looks like this (Figure~\ref{fig-lungcap_scatter4}):

\begin{figure}

\centering{

\includegraphics[width=6.25in,height=\textheight]{images/lungcap_scatter4.png}

}

\caption{\label{fig-lungcap_scatter4}The Scatter of Age and Lung
Capacity with the marginal density plots}

\end{figure}%

The above density plots (light blue histograms) show that the data are
approximately \textbf{normally distributed} for both \texttt{Age} and
\texttt{LungCap} (we have a large sample so the graphs are reliable).

Additionally, the points in the scatter plot seem to be scattered around
an invisible \textbf{line}. The scatter plot also shows that, in
general, older participants tend to have higher lung capacity
(\textbf{positive association}).

The Pearson's correlation coefficient can quantify the strength of this
linear association (alternative is Spearman's correlation coefficients).

~

\subsection{Applying the Pearson's correlation coefficient,
r}\label{applying-the-pearsons-correlation-coefficient-r}

Running correlation in Jamovi requires only a few steps once the data is
ready to go. In the top menu navigate to:

\includegraphics[width=4.45in,height=0.52in]{lab9_files/figure-latex/mermaid-figure-2.png}

as shown below in Figure~\ref{fig-lungcap_scatter5}.

\begin{figure}

\centering{

\includegraphics[width=0.75\textwidth,height=\textheight]{images/lungcap_scatter5.png}

}

\caption{\label{fig-lungcap_scatter5}In the menu at the top, choose
Analyses \textgreater{} Regression
\includegraphics[width=0.36458in,height=0.34375in]{images/reg_icon.png}
\textgreater{} Correlation Matrix.}

\end{figure}%

The \emph{Correlation Matrix dialogue box} opens
(Figure~\ref{fig-lungcap_scatter7}). Transfer both \texttt{Age} and
\texttt{LungCap} variables from the left-hand pane into the right-hand
pane by highlighting the variables and pressing the Arrow Button
(\includegraphics[width=0.23958in,height=0.21875in]{images/right_arrow.png}).
Additionally, from the \emph{Correlation Coefficients} choices we can
select between the following three options: Pearson's, Spearman, or
Kendall's coefficient. We keep the default choice of
``\textbf{Pearson}''. Finally, from \textbf{Additional Options} check
``\textbf{Flag significant correlations}'' and the ``\textbf{Confidence
Intervals}'' boxes. We will end up with the following screen:

\begin{figure}

\centering{

\includegraphics[width=0.65\textwidth,height=\textheight]{images/lungcap_scatter7.png}

}

\caption{\label{fig-lungcap_scatter7}The Correlation Matrix dialogue box
options. Drag and drop the \textbf{Age} and \textbf{LungCap} into the
right-hand pane. Check the boxes of interest in the additional options.}

\end{figure}%

The output table should look like the following
(Figure~\ref{fig-lungcap_cortable}):

\begin{figure}

\centering{

\includegraphics[width=0.5\textwidth,height=\textheight]{images/lungcap_cortable.png}

}

\caption{\label{fig-lungcap_cortable}The correlation matrix table.}

\end{figure}%

\begin{tcolorbox}[enhanced jigsaw, bottomrule=.15mm, title={Interpretation of the results}, toprule=.15mm, opacitybacktitle=0.6, coltitle=black, opacityback=0, rightrule=.15mm, leftrule=.75mm, colback=white, bottomtitle=1mm, toptitle=1mm, colframe=quarto-callout-note-color-frame, arc=.35mm, breakable, titlerule=0mm, left=2mm, colbacktitle=quarto-callout-note-color!10!white]

There is evidence of a \textbf{very strong, positive, linear}
association between \textbf{Age} and \textbf{Lung Capacity} (r= 0.82,
95\% CI: 0.79 to 0.84, p \textless{} 0.001) which is significant.

\end{tcolorbox}

\chapter{LAB X: Simple linear regression}\label{sec-lab10}

When we have finished this Lab, we should be able to:

\begin{tcolorbox}[enhanced jigsaw, bottomrule=.15mm, title={Learning objectives}, toprule=.15mm, opacitybacktitle=0.6, coltitle=black, opacityback=0, rightrule=.15mm, leftrule=.75mm, colback=white, bottomtitle=1mm, toptitle=1mm, colframe=quarto-callout-caution-color-frame, arc=.35mm, breakable, titlerule=0mm, left=2mm, colbacktitle=quarto-callout-caution-color!10!white]

\begin{itemize}
\tightlist
\item
  Understand the linear regression model
\item
  Explore how a factor (independent variable) affect a response
  (dependent) variable.
\item
  Interpret the results
\end{itemize}

\end{tcolorbox}

In this Lab, we will use the ``LungCapacity'' dataset.

\subsection{Opening the file}\label{opening-the-file-4}

Open the dataset named ``LungCapacity'' from the file tab in the menu:

\begin{figure}

\centering{

\includegraphics[width=0.75\textwidth,height=\textheight]{images/lungcap.png}

}

\caption{\label{fig-lungcap1}The LungCapacity dataset}

\end{figure}%

Double-click on the variable name \texttt{Age} and change the measure
type from nominal
\includegraphics[width=0.23958in,height=0.21875in]{images/nominal.png}
to continuous
\includegraphics[width=0.23958in,height=0.21875in]{images/continuous.png}.

\subsection{Research question}\label{research-question-4}

Let's say that we want to \textbf{model} the association between age (in
years) and lung capacity (in liters) for the sample of 725 participants
in a survey. In other words, we want to find the parameters of a
mathematical equation such as \(y = \alpha + \beta \cdot x\).

\subsection{Hypothesis Testsing}\label{hypothesis-testsing-1}

\begin{tcolorbox}[enhanced jigsaw, bottomrule=.15mm, title={Null hypothesis and alternative hypothesis}, toprule=.15mm, opacitybacktitle=0.6, coltitle=black, opacityback=0, rightrule=.15mm, leftrule=.75mm, colback=white, bottomtitle=1mm, toptitle=1mm, colframe=quarto-callout-note-color-frame, arc=.35mm, breakable, titlerule=0mm, left=2mm, colbacktitle=quarto-callout-note-color!10!white]

\begin{itemize}
\item
  \(H_{0}:\) the two variables are not linearly related. There is no
  effect between age and lung capacity (\(β = 0\)).
\item
  \(H_{1}:\) the two variables are linearly related. There is an effect
  between age and lung capacity (\(β \neq 0\)).
\end{itemize}

\end{tcolorbox}

\subsection{Scatter plot}\label{scatter-plot}

We start our analysis by creating the scatter plot of the response
variable \texttt{LungCap} and the explanatory variable \texttt{Age}.

\begin{figure}

\centering{

\includegraphics[width=0.65\textwidth,height=\textheight]{images/linear1.png}

}

\caption{\label{fig-linear1}The Scatter plot of Age and Lung Capacity}

\end{figure}%

There is a clear upward trend indicating that increase in \texttt{Age}
tends to coincide with increase in \texttt{LungCap}. Moreover, the trend
seems to be linear, so a straight line can capture the overall pattern.

\subsection{Linear regression}\label{linear-regression}

The process of fitting a linear regression model to the data involves
finding a straight line that can be considered as the \textbf{best
representation} of the overall association between age and lung
capacity.

To choose a line, we need to explain what we mean by the ``best
representation'' of the data. A ``best-fitting'' line refers to the line
that \textbf{minimizes} the \textbf{sum of squared residuals (RSS)}.
Therefore, we refer to the resulting model as the least-squares linear
regression model and to the corresponding line as the
\textbf{least-squares regression line}.

\subsection{Fit a simple linear regression
model}\label{fit-a-simple-linear-regression-model}

On the Jamovi top menu navigate to

\includegraphics[width=4.39in,height=0.52in]{lab10_files/figure-latex/mermaid-figure-1.png}

as shown below (Figure~\ref{fig-linear2}).

\begin{figure}

\centering{

\includegraphics[width=0.85\textwidth,height=\textheight]{images/linear2.png}

}

\caption{\label{fig-linear2}In the menu at the top, choose Analyses
\textgreater{} Regression \textgreater{} Linear Regression.}

\end{figure}%

The \emph{Linear Regression dialogue box} opens
(Figure~\ref{fig-linear3}). From the left-hand pane drag the variable
\texttt{LunCap} into the \emph{Dependent Variable} field and the
variable \texttt{Age} into the \emph{Covariates} field on the right-hand
side, as shown below:

\begin{figure}

\centering{

\includegraphics[width=0.65\textwidth,height=\textheight]{images/linear3.png}

}

\caption{\label{fig-linear3}The Linear Regression dialogue box options.
Drag and drop the LunCap into the Dependent Variable field and the Age
into the Covariates field.}

\end{figure}%

Additionally, from the \emph{Model Coefficients} section tick the box
``\textbf{Confidence interval}'' in \textbf{Estimate}
(Figure~\ref{fig-linear4}):

\begin{figure}

\centering{

\includegraphics[width=0.75\textwidth,height=\textheight]{images/linear4.png}

}

\caption{\label{fig-linear4}Check the Confidence interval box in the
Model Coefficients section.}

\end{figure}%

The output table with the model coefficients should look like the
following (Figure~\ref{fig-linear5}):

\begin{figure}

\centering{

\includegraphics[width=0.75\textwidth,height=\textheight]{images/linear5.png}

}

\caption{\label{fig-linear5}The model coefficients table.}

\end{figure}%

Now, let's find the model equation from the regression table in
Figure~\ref{fig-linear5}. In the \textbf{Estimate} column are the
intercept \(a=0.54\) and the slope \(b=0.26\) for \texttt{Age}. Thus,
the equation of the regression line becomes:

\[
\begin{aligned}
\widehat{y} &= a + b \cdot x\\
\widehat{\text{LungCap}} &= a + b \cdot\text{Age}\\
\widehat{\text{LungCap}}&= 0.54 + 0.26\cdot\text{Age}
\end{aligned}
\]

Finally, the quality of our simple linear model is presented in
Figure~\ref{fig-linear6}:

\begin{figure}

\centering{

\includegraphics[width=0.35\textwidth,height=\textheight]{images/linear6.png}

}

\caption{\label{fig-linear6}The coefficient of determination \(R^2\).}

\end{figure}%

In our example takes the value 0.67. It indicates that about 67\% of the
variation in lung capacity can be explained by the variation of the age.
In simple linear regression \(\sqrt{0.67} = 0.82\) which equals to the
Pearson's correlation coefficient, r.

~

\begin{tcolorbox}[enhanced jigsaw, bottomrule=.15mm, title={Interpretation of the results}, toprule=.15mm, opacitybacktitle=0.6, coltitle=black, opacityback=0, rightrule=.15mm, leftrule=.75mm, colback=white, bottomtitle=1mm, toptitle=1mm, colframe=quarto-callout-note-color-frame, arc=.35mm, breakable, titlerule=0mm, left=2mm, colbacktitle=quarto-callout-note-color!10!white]

The regression coefficient (b=0.26) of the \texttt{Age} is significantly
different from zero (p \textless{} 0.001) and indicates that there's
\textbf{on average an increase of 0.26 liters} in lung capacity for
\textbf{every 1 year increase in age}. Note that the \(95\%\)CI (0.24 to
0.27) \textbf{does not} include the hypothesized null value of zero for
the slope.

\end{tcolorbox}

\chapter{Data}\label{data-1}

~ ~

\begin{itemize}
\tightlist
\item
  \textbf{Dataset: rses}
\end{itemize}

\begin{verbatim}
file:///C:/Users/kboug/AppData/Local/Temp/Rtmpy02wTQ/file3168602afc4/widget31685d9639d.html screenshot completed
\end{verbatim}

\begin{figure}

\centering{

\includegraphics{data_files/figure-pdf/fig-rses-1.pdf}

}

\caption{\label{fig-rses}Table with raw data.}

\end{figure}%

~ ~

\begin{itemize}
\tightlist
\item
  \textbf{Dataset: creatinine}
\end{itemize}

\begin{verbatim}
file:///C:/Users/kboug/AppData/Local/Temp/Rtmpy02wTQ/file316854097b41/widget31683954595.html screenshot completed
\end{verbatim}

\begin{figure}

\centering{

\includegraphics{data_files/figure-pdf/fig-creatinine-1.pdf}

}

\caption{\label{fig-creatinine}Table with raw data.}

\end{figure}%

~ ~

\begin{itemize}
\tightlist
\item
  \textbf{Dataset: depression}
\end{itemize}

\begin{verbatim}
file:///C:/Users/kboug/AppData/Local/Temp/Rtmpy02wTQ/file31687972cef/widget3168705c68e6.html screenshot completed
\end{verbatim}

\begin{figure}

\centering{

\includegraphics{data_files/figure-pdf/fig-depression-1.pdf}

}

\caption{\label{fig-depression}Table with raw data.}

\end{figure}%

~ ~

\begin{itemize}
\tightlist
\item
  \textbf{Dataset: weight}
\end{itemize}

\begin{verbatim}
file:///C:/Users/kboug/AppData/Local/Temp/Rtmpy02wTQ/file31686caa74ec/widget316860e8335e.html screenshot completed
\end{verbatim}

\begin{figure}

\centering{

\includegraphics{data_files/figure-pdf/fig-weight-1.pdf}

}

\caption{\label{fig-weight}Table with raw data.}

\end{figure}%

~ ~

\begin{itemize}
\tightlist
\item
  \textbf{Dataset: dataDWL}
\end{itemize}

\begin{verbatim}
file:///C:/Users/kboug/AppData/Local/Temp/Rtmpy02wTQ/file31683a1b640a/widget316840577555.html screenshot completed
\end{verbatim}

\begin{figure}

\centering{

\includegraphics{data_files/figure-pdf/fig-dataDWL-1.pdf}

}

\caption{\label{fig-dataDWL}Table with raw data.}

\end{figure}%

~ ~

\begin{itemize}
\tightlist
\item
  \textbf{Dataset: prematurity}
\end{itemize}

\begin{verbatim}
file:///C:/Users/kboug/AppData/Local/Temp/Rtmpy02wTQ/file316810e27117/widget3168652a228c.html screenshot completed
\end{verbatim}

\begin{figure}

\centering{

\includegraphics{data_files/figure-pdf/fig-prematurity-1.pdf}

}

\caption{\label{fig-prematurity}Table with raw data.}

\end{figure}%

~ ~

\begin{itemize}
\tightlist
\item
  \textbf{Dataset: LungCapacity}
\end{itemize}

\begin{verbatim}
file:///C:/Users/kboug/AppData/Local/Temp/Rtmpy02wTQ/file316848154953/widget316838201753.html screenshot completed
\end{verbatim}

\begin{figure}

\centering{

\includegraphics{data_files/figure-pdf/fig-LungCapacity-1.pdf}

}

\caption{\label{fig-LungCapacity}Table with raw data.}

\end{figure}%

\chapter{Presentations}\label{presentations}

\begin{itemize}
\tightlist
\item
  \textbf{Yli 2024}
\end{itemize}

\begin{figure}[H]

{\centering \includegraphics[width=0.6\textwidth,height=4.16667in]{presentations/Yli-psy-300-2025-26.pdf}

}

\caption{Yli-psy-300-2025-26}

\end{figure}%

~ ~ ~

\begin{itemize}
\tightlist
\item
  \textbf{Lecture 1: Introduction}
\end{itemize}

\begin{figure}[H]

{\centering \includegraphics[width=0.6\textwidth,height=4.16667in]{presentations/psy-300-1_introduction.pdf}

}

\caption{psy-300-1\_introduction}

\end{figure}%

\begin{itemize}
\tightlist
\item
  \textbf{Lecture 2: Descriptive Statistics}
\end{itemize}

\begin{figure}[H]

{\centering \includegraphics[width=0.6\textwidth,height=4.16667in]{presentations/psy-300-2_descriptive.pdf}

}

\caption{psy-300-2\_descriptive}

\end{figure}%

~ ~ ~

\begin{itemize}
\tightlist
\item
  \textbf{Lecture 3: Probability Distributions}
\end{itemize}

\begin{figure}[H]

{\centering \includegraphics[width=0.6\textwidth,height=4.16667in]{presentations/psy-300-3_distributions.pdf}

}

\caption{psy-300-3\_distributions}

\end{figure}%

~ ~ ~

\begin{itemize}
\tightlist
\item
  \textbf{Lecture 4: Sampling Distribution and Confidence Interval}
\end{itemize}

\begin{figure}[H]

{\centering \includegraphics[width=0.6\textwidth,height=4.16667in]{presentations/psy-300-4_sampling_distribution.pdf}

}

\caption{psy-300-4\_sampling\_distribution}

\end{figure}%

~ ~ ~

\begin{itemize}
\tightlist
\item
  \textbf{Lecture 5: Hypothesis testing}
\end{itemize}

\begin{figure}[H]

{\centering \includegraphics[width=0.6\textwidth,height=4.16667in]{presentations/psy-300-5_hypothesis_testing.pdf}

}

\caption{psy-300-5\_hypothesis\_testing}

\end{figure}%

~ ~ ~

\begin{itemize}
\tightlist
\item
  \textbf{Lecture 6: Hypothesis testing for two samples}
\end{itemize}

\begin{figure}[H]

{\centering \includegraphics[width=0.6\textwidth,height=4.16667in]{presentations/psy-300-6_two_samples.pdf}

}

\caption{psy-300-6\_two\_samples}

\end{figure}%

~ ~ ~

\begin{itemize}
\tightlist
\item
  \textbf{Lecture 7: ANOVA}
\end{itemize}

\begin{figure}[H]

{\centering \includegraphics[width=0.6\textwidth,height=4.16667in]{presentations/psy-300-7_ANOVA.pdf}

}

\caption{psy-300-7\_ANOVA}

\end{figure}%

~ ~ ~

\begin{itemize}
\tightlist
\item
  \textbf{Lectures 8-9: Chi-square test}
\end{itemize}

\begin{figure}[H]

{\centering \includegraphics[width=0.6\textwidth,height=4.16667in]{presentations/psy-300-8_chi_square.pdf}

}

\caption{psy-300-8\_chi\_square}

\end{figure}%

~ ~ ~

\begin{itemize}
\tightlist
\item
  \textbf{Lecture 9-10: Correlation and Simple Linear Regression}
\end{itemize}

\begin{figure}[H]

{\centering \includegraphics[width=0.6\textwidth,height=4.16667in]{presentations/psy-300-9_10_correlation-linear.pdf}

}

\caption{psy-300-9\_10\_correlation-linear}

\end{figure}%

\bookmarksetup{startatroot}

\chapter*{References}\label{references}
\addcontentsline{toc}{chapter}{References}

\markboth{References}{References}

\phantomsection\label{refs}
\begin{CSLReferences}{1}{0}
\bibitem[\citeproctext]{ref-friligkou2024}
Friligkou, Eleni, Solveig Løkhammer, Brenda Cabrera-Mendoza, Jie Shen,
Jun He, Giovanni Deiana, Mihaela Diana Zanoaga, et al. 2024. {``Gene
Discovery and Biological Insights into Anxiety Disorders from a
Large-Scale Multi-Ancestry Genome-Wide Association Study.''}
\emph{Nature Genetics}, September.
\url{https://doi.org/10.1038/s41588-024-01908-2}.

\bibitem[\citeproctext]{ref-garcuxeda2019}
García, Jorge Acosta, Francisco Checa y Olmos, Manuel Lucas Matheu, and
Tesifón Parrón Carreño. 2019. {``Self Esteem Levels Vs Global Scores on
the Rosenberg Self-Esteem Scale.''} \emph{Heliyon} 5 (3): e01378.
\url{https://doi.org/10.1016/j.heliyon.2019.e01378}.

\bibitem[\citeproctext]{ref-grimes2002}
Grimes, David A, and Kenneth F Schulz. 2002. {``Bias and Causal
Associations in Observational Research.''} \emph{The Lancet} 359 (9302):
248--52. \url{https://doi.org/10.1016/s0140-6736(02)07451-2}.

\bibitem[\citeproctext]{ref-hyndman1996}
Hyndman, Rob J., and Yanan Fan. 1996. {``Sample Quantiles in Statistical
Packages.''} \emph{The American Statistician} 50 (4): 361--65.
\url{https://doi.org/10.1080/00031305.1996.10473566}.

\end{CSLReferences}




\end{document}
